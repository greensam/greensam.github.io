\documentclass[12p]{amsart}

\usepackage{etex}

% index
%\usepackage{makeidx}   % not needed in AMS styles
\makeindex
\newcommand{\indx}[1]{\index{#1}#1}
\renewcommand{\seename}{see}    % AMS style has it as "see also"

% fonts
\usepackage{palatino}
\usepackage{pxfonts}
\usepackage{pifont}
\usepackage{alltt}
\usepackage{textcomp}
\usepackage{upquote}

% color 
\usepackage[usenames,dvipsnames]{xcolor}

\newcommand{\highlightcolor}{BrickRed}
\newcommand{\lowlightcolor}{Gray}

% listings, links, etc. 
\usepackage[colorlinks=true,
            urlcolor=\highlightcolor,
            linkcolor=\highlightcolor,
            citecolor=\highlightcolor]{hyperref}
\urlstyle{same}  % don't use monospace font for urls 
\usepackage{listings}

% code listings using minted and pygments
\usepackage{minted}
% At some point, need to write a better pygment style as per http://pygments.org/docs/styles/
\usemintedstyle{bw} %trac
\setminted{xleftmargin=2\parindent}
% This environment definition doesn't work, since minted is a verbatim
% environment that needs to see a real \end{minted} not
% a \end{ocamlex}. Too bad.
%\newenvironment{ocamlex}{\begin{minted}{ocaml}}{\end{minted}}


% Configuration for Caml listings using camltex

\newlength{\abovecamlskip}
\setlength{\abovecamlskip}{\bigskipamount}
\newlength{\belowcamlskip}
\setlength{\belowcamlskip}{\bigskipamount}
\newlength{\camlblanklineskip}
\setlength{\camlblanklineskip}{\medskipamount}

% This command allows putting together multiple caml listings with
% controllable space between them.
\newcommand{\nextcaml}{%
\vspace{\camlblanklineskip}%
}

\newcommand{\ocaml}{\texttt{OCaml}}

% tables
\usepackage{tabulary}
\usepackage{booktabs}
\usepackage[tableposition=below]{caption}
\usepackage{graphicx}

% fancy verbatim environment, useful for typesetting program output
\usepackage{fancyvrb}

% linguistic examples
\usepackage{lingmacros}

% interline spacing
\usepackage{setspace}
\onehalfspacing
%\doublespacing

% em-dashes (some inspiration taken from ltugboat.sty)
\newcommand{\dashpunctskip}{\hspace{.25em}\relax}
%\newcommand{\dd}{\unskip{\,--\,}\ignorespaces}
%\newcommand{\dd}{\unskip{---}\ignorespaces}
%\newcommand{\dd}{\unskip{ -- }\ignorespaces}
\newcommand{\dd}{\unskip\nobreak\dashpunctskip--\dashpunctskip\ignorespaces}

% if using xelatex, the following synonym works
%\newcommand{\—}{\dd}

% a literal (like a filename) which should be presented verbatim, but
% allow line breaks but no hyphens inserted
\newcommand{\literal}[1]{\nolinkurl{#1}}
\usepackage{underscore}
\newcommand{\code}[1]{\texttt{#1}}

% tree drawing
%\usepackage{parsetree}
\usepackage{tikz}
\usepackage{tikz-qtree}

% document structure
\newcommand{\newsection}[1]{\clearpage\section{#1}}
\newcommand{\topic}[1]{\subsection*{#1}}
\setcounter{tocdepth}{1}

% symbols
\usepackage{amssymb}
\usepackage{bbding}
\usepackage{mathtools}
\usepackage{textcomp}
\usepackage{extarrows}  % extensible arrows

% random symbols
\renewcommand{\tilde}{\textasciitilde}

% sets and sequences
\newcommand{\seq}[1]{{\langle #1 \rangle}}
\newcommand{\set}[1]{{\{ #1 \}}}
\newcommand{\vect}[1]{\mathbf{#1}} % variables over strings and sequences
\renewcommand{\emptyset}{\varnothing}
\newcommand{\emptystr}{\varepsilon}
\newcommand{\generic}[1]{\mbox{$\langle$\textrm{\textit{#1}}$\rangle$}}

% floor and ceiling
\newcommand{\floor}[1]{\left\lfloor #1 \right\rfloor}
\newcommand{\ceil}[1]{\left\lceil #1 \right\rceil}

% guillemets
\usepackage[T1]{fontenc}

% operators
\newcommand{\argmax}{\operatornamewithlimits{argmax}}
\newcommand{\cnt}[1]{\sharp(#1)}
\newcommand{\definedby}{\triangleq}
\newcommand{\Prob}{{\Pr}}
\newcommand{\Prcap}{\widehat\Prob}
\DeclareMathOperator{\Perplex}{Pp}
\newcommand{\Pathprob}{{\Prob}_{path}} 
\newcommand{\blt}{\mathop{\bullet}}

% semiring operations and values
% circled plus and times for semirings
%\newcommand{\vplusinternal}{\ooalign{\hss$\circ$\hss\cr$+$}}
%\newcommand{\vtimesinternal}{\ooalign{\hss$\circ$\hss\cr$\times$}}
\newcommand{\vplus}{\mathrel{\mathchoice{\vplusinternal}{\vplusinternal}{\scriptsize\vplusinternal}{\tiny\vplusinternal}}}
\newcommand{\vtimes}{\mathrel{\mathchoice{\vtimesinternal}{\vtimesinternal}{\scriptsize\vtimesinternal}{\tiny\vtimesinternal}}}
%\newcommand{\vplus}{\oplus}
%\newcommand{\vtimes}{\otimes}
%\newcommand{\vzero}{{\ooalign{\hss$\circ$\hss\cr$0$}}}
%\newcommand{\vone}{{\ooalign{\hss$\circ$\hss\cr$
%\usepackage{fge}
%\newcommand{\vzero}{\fgestruckzero}
%\newcommand{\vone}{\fgestruckone}
%\newcommand{\vzero}{0}
%\newcommand{\vone}{1}

% numerals
\newcommand{\num}[1]{\underline{\ensuremath{#1}}}
\newcommand{\vzero}{\num{0}}
\newcommand{\vone}{\num{1}}
\newcommand{\vplusinternal}{\underline{+}}
\newcommand{\vtimesinternal}{\underline{\times}}

% logic
\newcommand{\Land}{\wedge}
\newcommand{\Lor}{\vee}
\newcommand{\Lnot}{\neg}
\newcommand{\Lif}{\rightharpoonup}
\newcommand{\Liff}{\rightleftharpoons}
\newcommand{\Ltrue}{T}
\newcommand{\Lfalse}{F}

% types
\newcommand{\cross}{\times}
%\newcommand{\to}{\rightarrow}

% lambda calculus
\newcommand{\subst}[3]{#1[#3 \mapsto #2]}   % subst P for x in Q is \subst{Q}{P}{x}
\newcommand{\rewrites}[1]{\xlongrightarrow{\ #1\ }}
\newcommand{\rwrts}{\rewrites{}}
\newcommand{\redex}[1]{\underline{#1}}

% OCaml rewriting
\newcommand{\oredex}[1]{\underline{#1}}
\newcommand{\orwrts}{\(\rwrts\)}

% better spacing in conditional probabilities, instead of
% \newcommand{\given}{|}           % too little space
%\newcommand{\given}{\mid}     % too much space
\newcommand{\given}{\,|\,}      % just right

% values of random variables
\newcommand{\rvval}[1]{\mathsf{#1}}

% indicator functions
\newcommand{\indic}[1]{\mathbb{I}\left[#1\right]}

% start and end of string symbols
\newcommand{\strbeg}{\mbox{\guillemotleft}}
\newcommand{\strend}{\mbox{\guillemotright}}

% different types of symbols
\newcommand{\sym}[1]{\mbox{#1}}
% nonterminals and terminals
\newcommand{\ntm}[1]{\mbox{\textit{#1}}}
\newcommand{\tm}[1]{\mbox{\textrm{#1}}}
\newcommand{\term}[1]{\ntm{#1}} % deprecated; use \nt
\newcommand{\func}[1]{\operatorname{\mathit{#1}}}

% grammar rules
\newcommand{\goesto}{\rightarrow}

% punctuation in equations
\newcommand{\eqpunc}[1]{{\makebox[0pt][l]{\qquad\rm{#1}}}}

% first mentions and other marginalia
\newcommand{\firstusefont}[1]{\textsc{#1}}
\newcommand{\firstuse}[1]{\firstusemarked{#1}{#1}}
% first use with separate margin and inline (= index) forms
\newcommand{\firstusemarked}[2]{\firstusemarkedindexed{#1}{#2}{#1}}
% first use with separate margin, inline, and index forms
%   for placing the index item at a separate location in the index, use 
%   \firstusemarked{margin}{inline}{index@indexloc}, e.g.,
%   \firstusemarked{$n$-gram}{$n$-gram}{n-gram@$n$-gram}
\newcommand{\firstusemarkedindexed}[3]{\index{#3}\firstusefont{#2}%
  \marginpar[\begin{spacing}{.75}\vspace{-.75ex}\raggedleft\scriptsize\firstusefont{\raggedleft\color{\highlightcolor} #1}\end{spacing}\vspace*{2.75ex}]{\begin{spacing}{.75}\vspace{-.75ex}\raggedright\scriptsize\firstusefont{\raggedright\color{\highlightcolor} #1}\end{spacing}\vspace*{2.75ex}}}


% instructor material
\newif\ifinstructor
\instructortrue
\instructorfalse

\newcommand{\instructor}[1]{\ifinstructor\begin{instructorchunk}#1\end{instructorchunk}\fi}
\newenvironment{instructorchunk}%
{\marginpar[\raggedleft\color{red}\PencilRight]{\raggedright\color{red}\PencilLeft}\color{red}}{}

\newcommand{\inclass}[2]{\ifinstructor\begin{instructorchunk}\begin{description}\item[Question]#1\par\item[Answer]#2\end{description}\end{instructorchunk}\fi}


% displaying shell commands and interactive python

\DefineVerbatimEnvironment%
{commandblock}{Verbatim}
{commandchars=\\\{\}, xleftmargin=2\parindent}

\newcommand{\command}[1]{%
\begin{quote}
\begin{alltt}
#1
\end{alltt}
\end{quote}}
\newcommand{\shellcommand}[1]{\command{\$ #1}}

% advanced topics

\usepackage{manfnt}
%\newcommand{\advanced}[1]{\footnote{\manimpossiblecube: #1}}
%\newcommand{\advanced}[1]{\begin{quote}\color{\lowlightcolor}\manimpossiblecube~\small
%    #1\end{quote}}

%\newenvironment{advanced}{\begin{quote}\color{\lowlightcolor}\manimpossiblecube~\small}{\end{quote}}

\newenvironment{advanced}{\begin{list}{}{%
      \setlength{\leftmargin}{15ex}%
      \setlength{\rightmargin}{0ex}%
    }
  \item \color{\highlightcolor}\manimpossiblecube\hspace{2ex}\small \color{\lowlightcolor} }{\end{list}}

%\renewenvironment{advanced}{\par\begingroup\color{gray}\makebox[0pt][r]{\manimpossiblecube\hspace{5ex}}\small}{\endgroup\par}

% bibliography
\usepackage[colon]{natbib}

% theorems
\newtheorem{theorem}{Theorem}

% problems and exercises
\newcommand{\probmark}{\hfill{$\Box$}}
\newtheorem{probleminternal}{Problem}
\newtheorem{questioninternal}{Question}
\newtheorem{exerciseinternal}[probleminternal]{Exercise}% 
\newtheorem{stageinternal}[probleminternal]{Stage}% 
\newenvironment{exercise}{\begin{exerciseinternal}}{\probmark\end{exerciseinternal}}
\newenvironment{problem}{\begin{probleminternal}}{\probmark\end{probleminternal}}
\newenvironment{stage}{\begin{stageinternal}}{\probmark\end{stageinternal}}
\newenvironment{question}{\begin{questioninternal}}{\probmark\end{questioninternal}}
\newtheorem{definition}{Definition}

\newcommand{\nextproblem}{\vspace{2em}}

\newcommand{\solution}[1]{{\color{\highlightcolor}
   \begin{quote}\textbf{Solution:} #1\end{quote}
   }}
 
% temporary notes
\newcommand{\note}[1]{{\color{red}{[[[#1]]]}}}
%\newcommand{\note}[1]{{\color{red} \textbf{[[[#1]]]}}}
\renewcommand{\note}[1]{}

\sloppy

% line numbering
% Don't need lineno since minted loads it.
%\usepackage[switch*,mathlines]{lineno}
\runningpagewiselinenumbers
\setpagewiselinenumbers
\switchlinenumbers*
\linenumberdisplaymath
\modulolinenumbers[5]
\renewcommand\linenumberfont{\normalfont\tiny\sffamily\color{Tan}}

\usepackage{bussproofs}
\newcommand{\SideCond}[1]{\RightLabel{\ #1}}

\newcommand{\types}{\mathcal{T}}
\newcommand{\consts}[1]{\mathcal{C}_{#1}}
\newcommand{\vars}[1]{\mathcal{X}_{#1}}
\newcommand{\fto}[2]{\langle #1, #2 \rangle}
\newcommand{\domain}[2]{\mathcal{D}_{#1,#2}}
\newcommand{\true}{\mathsf{True}}
\newcommand{\false}{\mathsf{False}}
\newcommand{\model}{\mathcal{M}}
\newcommand{\assign}{\mathcal{G}}
\newcommand{\dennoparams}[1]{\llbracket #1 \rrbracket}
\newcommand{\den}[3]{\dennoparams{#1}_{#2,#3}}
\newcommand{\denmg}[1]{\den{#1}{\model}{\assign}}



\title[]{CS 51\\ Code Review 3}
\date{\today}
\author{Samuel Green and Gabbi Merz}


\usepackage{fancyvrb,color}

\makeatletter
\def\PY@reset{\let\PY@it=\relax \let\PY@bf=\relax%
    \let\PY@ul=\relax \let\PY@tc=\relax%
    \let\PY@bc=\relax \let\PY@ff=\relax}
\def\PY@tok#1{\csname PY@tok@#1\endcsname}
\def\PY@toks#1+{\ifx\relax#1\empty\else%
    \PY@tok{#1}\expandafter\PY@toks\fi}
\def\PY@do#1{\PY@bc{\PY@tc{\PY@ul{%
    \PY@it{\PY@bf{\PY@ff{#1}}}}}}}
\def\PY#1#2{\PY@reset\PY@toks#1+\relax+\PY@do{#2}}

\expandafter\def\csname PY@tok@ch\endcsname{\let\PY@it=\textit\def\PY@tc##1{\textcolor[rgb]{1.00,0.00,0.00}{##1}}}
\expandafter\def\csname PY@tok@cm\endcsname{\let\PY@it=\textit\def\PY@tc##1{\textcolor[rgb]{1.00,0.00,0.00}{##1}}}
\expandafter\def\csname PY@tok@cs\endcsname{\let\PY@it=\textit\def\PY@tc##1{\textcolor[rgb]{1.00,0.00,0.00}{##1}}}
\expandafter\def\csname PY@tok@cp\endcsname{\let\PY@it=\textit\def\PY@tc##1{\textcolor[rgb]{1.00,0.00,0.00}{##1}}}
\expandafter\def\csname PY@tok@s2\endcsname{\def\PY@tc##1{\textcolor[rgb]{0.00,0.61,0.00}{##1}}}
\expandafter\def\csname PY@tok@s1\endcsname{\def\PY@tc##1{\textcolor[rgb]{0.00,0.61,0.00}{##1}}}
\expandafter\def\csname PY@tok@nc\endcsname{\def\PY@tc##1{\textcolor[rgb]{0.00,0.46,0.46}{##1}}}
\expandafter\def\csname PY@tok@nd\endcsname{\def\PY@tc##1{\textcolor[rgb]{0.80,0.00,0.64}{##1}}}
\expandafter\def\csname PY@tok@si\endcsname{\def\PY@tc##1{\textcolor[rgb]{0.00,0.61,0.00}{##1}}}
\expandafter\def\csname PY@tok@nf\endcsname{\def\PY@tc##1{\textcolor[rgb]{0.76,0.31,0.00}{##1}}}
\expandafter\def\csname PY@tok@sh\endcsname{\def\PY@tc##1{\textcolor[rgb]{0.00,0.61,0.00}{##1}}}
\expandafter\def\csname PY@tok@c1\endcsname{\let\PY@it=\textit\def\PY@tc##1{\textcolor[rgb]{1.00,0.00,0.00}{##1}}}
\expandafter\def\csname PY@tok@kc\endcsname{\def\PY@tc##1{\textcolor[rgb]{0.00,0.00,1.00}{##1}}}
\expandafter\def\csname PY@tok@c\endcsname{\let\PY@it=\textit\def\PY@tc##1{\textcolor[rgb]{1.00,0.00,0.00}{##1}}}
\expandafter\def\csname PY@tok@sx\endcsname{\def\PY@tc##1{\textcolor[rgb]{0.00,0.61,0.00}{##1}}}
\expandafter\def\csname PY@tok@kd\endcsname{\def\PY@tc##1{\textcolor[rgb]{0.00,0.00,1.00}{##1}}}
\expandafter\def\csname PY@tok@ss\endcsname{\def\PY@tc##1{\textcolor[rgb]{0.00,0.61,0.00}{##1}}}
\expandafter\def\csname PY@tok@sr\endcsname{\def\PY@tc##1{\textcolor[rgb]{0.00,0.61,0.00}{##1}}}
\expandafter\def\csname PY@tok@k\endcsname{\def\PY@tc##1{\textcolor[rgb]{0.00,0.00,1.00}{##1}}}
\expandafter\def\csname PY@tok@kn\endcsname{\def\PY@tc##1{\textcolor[rgb]{0.00,0.00,1.00}{##1}}}
\expandafter\def\csname PY@tok@cpf\endcsname{\let\PY@it=\textit\def\PY@tc##1{\textcolor[rgb]{1.00,0.00,0.00}{##1}}}
\expandafter\def\csname PY@tok@kr\endcsname{\def\PY@tc##1{\textcolor[rgb]{0.00,0.00,1.00}{##1}}}
\expandafter\def\csname PY@tok@s\endcsname{\def\PY@tc##1{\textcolor[rgb]{0.00,0.61,0.00}{##1}}}
\expandafter\def\csname PY@tok@kp\endcsname{\def\PY@tc##1{\textcolor[rgb]{0.00,0.00,1.00}{##1}}}
\expandafter\def\csname PY@tok@kt\endcsname{\def\PY@tc##1{\textcolor[rgb]{0.00,0.00,1.00}{##1}}}
\expandafter\def\csname PY@tok@sc\endcsname{\def\PY@tc##1{\textcolor[rgb]{0.00,0.61,0.00}{##1}}}
\expandafter\def\csname PY@tok@sb\endcsname{\def\PY@tc##1{\textcolor[rgb]{0.00,0.61,0.00}{##1}}}
\expandafter\def\csname PY@tok@se\endcsname{\def\PY@tc##1{\textcolor[rgb]{0.00,0.61,0.00}{##1}}}
\expandafter\def\csname PY@tok@sd\endcsname{\def\PY@tc##1{\textcolor[rgb]{0.00,0.61,0.00}{##1}}}

\def\PYZbs{\char`\\}
\def\PYZus{\char`\_}
\def\PYZob{\char`\{}
\def\PYZcb{\char`\}}
\def\PYZca{\char`\^}
\def\PYZam{\char`\&}
\def\PYZlt{\char`\<}
\def\PYZgt{\char`\>}
\def\PYZsh{\char`\#}
\def\PYZpc{\char`\%}
\def\PYZdl{\char`\$}
\def\PYZhy{\char`\-}
\def\PYZsq{\char`\'}
\def\PYZdq{\char`\"}
\def\PYZti{\char`\~}
% for compatibility with earlier versions
\def\PYZat{@}
\def\PYZlb{[}
\def\PYZrb{]}
\makeatother
\begin{document}
\maketitle
\tableofcontents

% \section{Topics}

% \begin{itemize}
%   \item Algebraic Data Types and Records
%   \item Modeling and Type Composition
%   \item Invariants
%   \item Design Choices
% \end{itemize}

Note that there are \textit{no} new syntax here -- we're just putting previous concepts
and syntax together and thinking about modeling tradeoffs and other
abstract concepts!

\section{Invariants}

\begin{definition}
An \textbf{invariant} is a condition that is assumed
to true of any value of a particular type, but that
is not necessarily enforced by the compiler. 
\end{definition} 

Lab 3 and Problem Set 3 focus significantly on 
the design and enforcement of invariants as a useful
mechanism for programming in OCaml.

What was the invariant of the \texttt{RGB} variant
of the \texttt{color} type in Lab 3? 

\ifsoln
\solution{The invariant was that the integer values
in the Cartesian type were all in the range (0, 255).}
\fi

Did the choice of the type help us enforce the invariant?
What burden does this place on us as programmers? Could
a different type definition have helped? Does the
inclusion of \texttt{color_label} help for any
specific reason? 

\ifsoln
\solution{The type system unfortunately can't do anything
more than guarantee that each type is an \texttt{int},
which is better than nothing. 

We had to write our own
validation code to \texttt{enforce} this
invariant, to make sure that it was always true. 
As programmers, this is work for us and for our 
collaborators and particularly emphasizes the importance
of code that is both self-documenting (well styled) and 
documented (i.e. invariants are recorded in a comment.)

The inclusion of \texttt{color_label} is a nice choice
because it covers many of the simplest values, meaning
that users of the code can construct common values of
our \texttt{color} type without having enforce the
invariant on their own. It's also maintainable, since
we can make changes to which red or green we want to use
by default very easily! }
\fi

\newpage
\section{Modeling Choices}

Because we have to autograde the labs and problem sets,
there doesn't end up being a lot of space to exercise
your modeling muscles when it comes to representing data
in OCaml. Did you agree with the modeling choices we made
in the lab? 

Consider for example the following type for color:

\begin{Verbatim}[commandchars=\\\{\}]
\PY{o}{\PYZsh{}} \PY{k}{type} \PY{n}{color\PYZus{}label} \PY{o}{=} \PY{n+nc}{Orange} \PY{o}{|} \PY{n+nc}{Brown} \PY{o}{|} \PY{n+nc}{Green} \PY{o}{;;}
\PY{k}{type} \PY{n}{color\PYZus{}label} \PY{o}{=} \PY{n+nc}{Orange} \PY{o}{|} \PY{n+nc}{Brown} \PY{o}{|} \PY{n+nc}{Green}
\PY{o}{\PYZsh{}} \PY{k}{type} \PY{n}{color} \PY{o}{=} \PY{n+nc}{Simple} \PY{k}{of} \PY{n}{color\PYZus{}label} \PY{o}{|} \PY{n+nc}{RGB} \PY{k}{of} \PY{k+kt}{int} \PY{o}{*} \PY{k+kt}{int} \PY{o}{*} \PY{k+kt}{int} \PY{o}{;;}
\PY{k}{type} \PY{n}{color} \PY{o}{=} \PY{n+nc}{Simple} \PY{k}{of} \PY{n}{color\PYZus{}label} \PY{o}{|} \PY{n+nc}{RGB} \PY{k}{of} \PY{k+kt}{int} \PY{o}{*} \PY{k+kt}{int} \PY{o}{*} \PY{k+kt}{int}
\end{Verbatim}

Why did choose this definition, particularly of \texttt{RGB}? What
do you think of this redefinition? 

\begin{Verbatim}[commandchars=\\\{\}]
\PY{o}{\PYZsh{}} \PY{k}{type} \PY{n}{rgb} \PY{o}{=} \PY{o}{\PYZob{}} \PY{n}{r} \PY{o}{:} \PY{k+kt}{int} \PY{o}{;} \PY{n}{g} \PY{o}{:} \PY{k+kt}{int} \PY{o}{;} \PY{n}{b} \PY{o}{:} \PY{k+kt}{int} \PY{o}{\PYZcb{}} \PY{o}{;;} 
\PY{k}{type} \PY{n}{rgb} \PY{o}{=} \PY{o}{\PYZob{}} \PY{n}{r} \PY{o}{:} \PY{k+kt}{int}\PY{o}{;} \PY{n}{g} \PY{o}{:} \PY{k+kt}{int}\PY{o}{;} \PY{n}{b} \PY{o}{:} \PY{k+kt}{int}\PY{o}{;} \PY{o}{\PYZcb{}}
\PY{o}{\PYZsh{}} \PY{k}{type} \PY{n}{color} \PY{o}{=} \PY{o}{\PYZob{}} \PY{n}{simple} \PY{o}{:} \PY{n}{color\PYZus{}label} \PY{n}{option}\PY{o}{;} \PY{n}{rgb} \PY{o}{:} \PY{n}{rgb} \PY{n}{option} \PY{o}{\PYZcb{}} \PY{o}{;;}
\PY{k}{type} \PY{n}{color} \PY{o}{=} \PY{o}{\PYZob{}} \PY{n}{simple} \PY{o}{:} \PY{n}{color\PYZus{}label} \PY{n}{option}\PY{o}{;} \PY{n}{rgb} \PY{o}{:} \PY{n}{rgb} \PY{n}{option}\PY{o}{;} \PY{o}{\PYZcb{}}
\end{Verbatim}

Is this example better or worse? Is there something better 
or worse? Is there an example that's similarly ``good'' to 
the first version but that uses a record type instead? 

\ifsoln
\solution{There's a reasonable argument that representing
an RGB triple as a record could make sense. }
\begin{Verbatim}[commandchars=\\\{\}]
\PY{o}{\PYZsh{}} \PY{k}{type} \PY{n}{rgb} \PY{o}{=} \PY{o}{\PYZob{}} \PY{n}{r} \PY{o}{:} \PY{k+kt}{int} \PY{o}{;} \PY{n}{g} \PY{o}{:} \PY{k+kt}{int} \PY{o}{;} \PY{n}{b} \PY{o}{:} \PY{k+kt}{int} \PY{o}{\PYZcb{}} \PY{o}{;;}
\PY{k}{type} \PY{n}{rgb} \PY{o}{=} \PY{o}{\PYZob{}} \PY{n}{r} \PY{o}{:} \PY{k+kt}{int}\PY{o}{;} \PY{n}{g} \PY{o}{:} \PY{k+kt}{int}\PY{o}{;} \PY{n}{b} \PY{o}{:} \PY{k+kt}{int}\PY{o}{;} \PY{o}{\PYZcb{}}
\PY{o}{\PYZsh{}} \PY{k}{type} \PY{n}{color} \PY{o}{=} \PY{n+nc}{Simple} \PY{k}{of} \PY{n}{color\PYZus{}label} \PY{o}{|} \PY{n+nc}{RGB} \PY{k}{of} \PY{n}{rgb} \PY{o}{;;}
\PY{k}{type} \PY{n}{color} \PY{o}{=} \PY{n+nc}{Simple} \PY{k}{of} \PY{n}{color\PYZus{}label} \PY{o}{|} \PY{n+nc}{RGB} \PY{k}{of} \PY{n}{rgb}
\end{Verbatim}
\fi

\newpage
\section{Concise and Field Punning: \texttt{valid_date}}

Here's a long version of the \texttt{valid_date} function.
What do you think of its design and style? 


\begin{Verbatim}[commandchars=\\\{\}]
\PY{o}{\PYZsh{}} \PY{k}{let} \PY{n}{valid\PYZus{}date} \PY{o}{(}\PY{n}{d} \PY{o}{:} \PY{n}{date}\PY{o}{)} \PY{o}{:} \PY{n}{date} \PY{o}{=}
  \PY{k}{if} \PY{n}{d}\PY{o}{.}\PY{n}{year} \PY{o}{\PYZlt{}}\PY{o}{=} \PY{l+m+mi}{0} \PY{k}{then} \PY{k}{raise} \PY{o}{(}\PY{n+nc}{Invalid\PYZus{}Date} \PY{l+s+s2}{\PYZdq{}}\PY{l+s+s2}{only positive years}\PY{l+s+s2}{\PYZdq{}}\PY{o}{)} \PY{k}{else}
  \PY{k}{if} \PY{n}{d}\PY{o}{.}\PY{n}{month} \PY{o}{=} \PY{l+m+mi}{1} \PY{o}{|}\PY{o}{|} \PY{n}{d}\PY{o}{.}\PY{n}{month} \PY{o}{=} \PY{l+m+mi}{3} \PY{o}{|}\PY{o}{|} \PY{n}{d}\PY{o}{.}\PY{n}{month} \PY{o}{=} \PY{l+m+mi}{5} \PY{o}{|}\PY{o}{|} \PY{n}{d}\PY{o}{.}\PY{n}{month} \PY{o}{=} \PY{l+m+mi}{7} \PY{o}{|}\PY{o}{|} \PY{n}{d}\PY{o}{.}\PY{n}{month} \PY{o}{=} \PY{l+m+mi}{8} \PY{o}{|}\PY{o}{|}
     \PY{n}{d}\PY{o}{.}\PY{n}{month} \PY{o}{=} \PY{l+m+mi}{10} \PY{o}{|}\PY{o}{|} \PY{n}{d}\PY{o}{.}\PY{n}{month} \PY{o}{=} \PY{l+m+mi}{12} \PY{k}{then}
    \PY{o}{(}\PY{k}{if} \PY{n}{d}\PY{o}{.}\PY{n}{day} \PY{o}{\PYZgt{}} \PY{l+m+mi}{31} \PY{k}{then} \PY{k}{raise} \PY{o}{(}\PY{n+nc}{Invalid\PYZus{}Date} \PY{l+s+s2}{\PYZdq{}}\PY{l+s+s2}{too many days}\PY{l+s+s2}{\PYZdq{}}\PY{o}{)} \PY{k}{else}
    \PY{k}{if} \PY{n}{d}\PY{o}{.}\PY{n}{day} \PY{o}{\PYZlt{}} \PY{l+m+mi}{1} \PY{k}{then} \PY{k}{raise} \PY{o}{(}\PY{n+nc}{Invalid\PYZus{}Date} \PY{l+s+s2}{\PYZdq{}}\PY{l+s+s2}{days must be \PYZgt{}1}\PY{l+s+s2}{\PYZdq{}}\PY{o}{)} \PY{k}{else}
    \PY{n}{d}\PY{o}{)} \PY{k}{else}
  \PY{k}{if} \PY{n}{d}\PY{o}{.}\PY{n}{month} \PY{o}{=} \PY{l+m+mi}{4} \PY{o}{|}\PY{o}{|} \PY{n}{d}\PY{o}{.}\PY{n}{month} \PY{o}{=} \PY{l+m+mi}{6} \PY{o}{|}\PY{o}{|} \PY{n}{d}\PY{o}{.}\PY{n}{month} \PY{o}{=} \PY{l+m+mi}{9} \PY{o}{|}\PY{o}{|} \PY{n}{d}\PY{o}{.}\PY{n}{month} \PY{o}{=} \PY{l+m+mi}{11} \PY{k}{then}
    \PY{o}{(}\PY{k}{if} \PY{n}{d}\PY{o}{.}\PY{n}{day} \PY{o}{\PYZgt{}} \PY{l+m+mi}{30} \PY{k}{then} \PY{k}{raise} \PY{o}{(}\PY{n+nc}{Invalid\PYZus{}Date} \PY{l+s+s2}{\PYZdq{}}\PY{l+s+s2}{too many days}\PY{l+s+s2}{\PYZdq{}}\PY{o}{)} \PY{k}{else}
    \PY{k}{if} \PY{n}{d}\PY{o}{.}\PY{n}{day} \PY{o}{\PYZlt{}} \PY{l+m+mi}{1} \PY{k}{then} \PY{k}{raise} \PY{o}{(}\PY{n+nc}{Invalid\PYZus{}Date} \PY{l+s+s2}{\PYZdq{}}\PY{l+s+s2}{days must be \PYZgt{}1}\PY{l+s+s2}{\PYZdq{}}\PY{o}{)} \PY{k}{else}
    \PY{n}{d}\PY{o}{)} \PY{k}{else}
  \PY{k}{if} \PY{n}{d}\PY{o}{.}\PY{n}{month} \PY{o}{=} \PY{l+m+mi}{2} \PY{k}{then}
    \PY{o}{(}\PY{k}{if} \PY{n}{d}\PY{o}{.}\PY{n}{year} \PY{o+ow}{mod} \PY{l+m+mi}{4} \PY{o}{=} \PY{l+m+mi}{0} \PY{o}{\PYZam{}\PYZam{}} \PY{n}{d}\PY{o}{.}\PY{n}{year} \PY{o+ow}{mod} \PY{l+m+mi}{100} \PY{o}{\PYZlt{}}\PY{o}{\PYZgt{}} \PY{l+m+mi}{0} \PY{o}{|}\PY{o}{|} \PY{n}{d}\PY{o}{.}\PY{n}{year} \PY{o+ow}{mod} \PY{l+m+mi}{400} \PY{o}{=} \PY{l+m+mi}{0} \PY{k}{then}
      \PY{k}{if} \PY{n}{d}\PY{o}{.}\PY{n}{day} \PY{o}{\PYZgt{}} \PY{l+m+mi}{29} \PY{k}{then} \PY{k}{raise} \PY{o}{(}\PY{n+nc}{Invalid\PYZus{}Date} \PY{l+s+s2}{\PYZdq{}}\PY{l+s+s2}{too many days}\PY{l+s+s2}{\PYZdq{}}\PY{o}{)} \PY{k}{else}
      \PY{k}{if} \PY{n}{d}\PY{o}{.}\PY{n}{day} \PY{o}{\PYZlt{}} \PY{l+m+mi}{1} \PY{k}{then} \PY{k}{raise} \PY{o}{(}\PY{n+nc}{Invalid\PYZus{}Date} \PY{l+s+s2}{\PYZdq{}}\PY{l+s+s2}{days must be \PYZgt{}1}\PY{l+s+s2}{\PYZdq{}}\PY{o}{)} \PY{k}{else}
      \PY{n}{d}
    \PY{k}{else} 
      \PY{k}{if} \PY{n}{d}\PY{o}{.}\PY{n}{day} \PY{o}{\PYZgt{}} \PY{l+m+mi}{28} \PY{k}{then} \PY{k}{raise} \PY{o}{(}\PY{n+nc}{Invalid\PYZus{}Date} \PY{l+s+s2}{\PYZdq{}}\PY{l+s+s2}{too many days}\PY{l+s+s2}{\PYZdq{}}\PY{o}{)} \PY{k}{else}
      \PY{k}{if} \PY{n}{d}\PY{o}{.}\PY{n}{day} \PY{o}{\PYZlt{}} \PY{l+m+mi}{1} \PY{k}{then} \PY{k}{raise} \PY{o}{(}\PY{n+nc}{Invalid\PYZus{}Date} \PY{l+s+s2}{\PYZdq{}}\PY{l+s+s2}{days must be \PYZgt{}1}\PY{l+s+s2}{\PYZdq{}}\PY{o}{)} \PY{k}{else}
      \PY{n}{d}\PY{o}{)} \PY{k}{else}
  \PY{k}{raise} \PY{o}{(}\PY{n+nc}{Invalid\PYZus{}Date} \PY{l+s+s2}{\PYZdq{}}\PY{l+s+s2}{bad month}\PY{l+s+s2}{\PYZdq{}}\PY{o}{)} \PY{o}{;;}
\end{Verbatim}

What are some first observations about this code, perhaps some
easy simplifications that could be made? 

\ifsoln
\solution{Here's a non-exhaustive list of things that come to mind.}
\begin{itemize}
  \item Day, Month, and Year are \textit{all} referenced multiple times
        throughout this piece of code. Using \textbf{field punning} would
        make a lot of sense here.
  \item Exception raising logic is repeated -- we could extract the
        raising logic into a separate helper function. 
  \item This is a situation where a \texttt{match} makes sense for an 
        \texttt{int}, because of there are distinct cases for us to
        consider. 
\end{itemize}

Here's a potential improved version:
\begin{Verbatim}[commandchars=\\\{\}]
\PY{o}{\PYZsh{}} \PY{k}{let} \PY{n}{valid\PYZus{}date} \PY{o}{(}\PY{o}{\PYZob{}} \PY{n}{year}\PY{o}{;} \PY{n}{month}\PY{o}{;} \PY{n}{day} \PY{o}{\PYZcb{}} \PY{k}{as} \PY{n}{d}\PY{o}{)} \PY{o}{:} \PY{n}{date} \PY{o}{=} 
  \PY{k}{let} \PY{n}{raise\PYZus{}bad} \PY{n}{m} \PY{o}{=} \PY{k}{raise} \PY{o}{(}\PY{n+nc}{Invalid\PYZus{}Date} \PY{n}{m}\PY{o}{)} \PY{k}{in}
  \PY{k}{if} \PY{n}{year} \PY{o}{\PYZlt{}} \PY{l+m+mi}{0} \PY{k}{then} \PY{n}{raise\PYZus{}bad} \PY{l+s+s2}{\PYZdq{}}\PY{l+s+s2}{only positive years}\PY{l+s+s2}{\PYZdq{}} 
  \PY{k}{else}
    \PY{k}{let} \PY{n}{leap} \PY{o}{=} \PY{n}{year} \PY{o+ow}{mod} \PY{l+m+mi}{4} \PY{o}{=} \PY{l+m+mi}{0} \PY{o}{\PYZam{}\PYZam{}} \PY{n}{year} \PY{o+ow}{mod} \PY{l+m+mi}{100} \PY{o}{\PYZlt{}}\PY{o}{\PYZgt{}} \PY{l+m+mi}{0} \PY{o}{|}\PY{o}{|} \PY{n}{year} \PY{o+ow}{mod} \PY{l+m+mi}{400} \PY{o}{=} \PY{l+m+mi}{0} \PY{k}{in}
    \PY{k}{let} \PY{n}{max\PYZus{}days} \PY{o}{=}
      \PY{k}{match} \PY{n}{month} \PY{k}{with}
      \PY{o}{|} \PY{l+m+mi}{1} \PY{o}{|} \PY{l+m+mi}{3} \PY{o}{|} \PY{l+m+mi}{5} \PY{o}{|} \PY{l+m+mi}{7} \PY{o}{|} \PY{l+m+mi}{8} \PY{o}{|} \PY{l+m+mi}{10} \PY{o}{|} \PY{l+m+mi}{12} \PY{o}{\PYZhy{}\PYZgt{}} \PY{l+m+mi}{31}
      \PY{o}{|} \PY{l+m+mi}{4} \PY{o}{|} \PY{l+m+mi}{6} \PY{o}{|} \PY{l+m+mi}{9} \PY{o}{|} \PY{l+m+mi}{11} \PY{o}{\PYZhy{}\PYZgt{}} \PY{l+m+mi}{30}
      \PY{o}{|} \PY{l+m+mi}{2} \PY{o}{\PYZhy{}\PYZgt{}} \PY{k}{if} \PY{n}{leap} \PY{k}{then} \PY{l+m+mi}{29} \PY{k}{else} \PY{l+m+mi}{28} 
      \PY{o}{|} \PY{o}{\PYZus{}} \PY{o}{\PYZhy{}\PYZgt{}} \PY{n}{raise\PYZus{}bad} \PY{l+s+s2}{\PYZdq{}}\PY{l+s+s2}{bad month}\PY{l+s+s2}{\PYZdq{}} \PY{k}{in}
    \PY{k}{if} \PY{n}{day} \PY{o}{\PYZgt{}} \PY{n}{max\PYZus{}days} \PY{k}{then} \PY{n}{raise\PYZus{}bad} \PY{l+s+s2}{\PYZdq{}}\PY{l+s+s2}{too many days}\PY{l+s+s2}{\PYZdq{}}
    \PY{k}{else} \PY{k}{if} \PY{n}{day} \PY{o}{\PYZlt{}} \PY{l+m+mi}{1} \PY{k}{then} \PY{n}{raise\PYZus{}bad} \PY{l+s+s2}{\PYZdq{}}\PY{l+s+s2}{days must be \PYZgt{}1}\PY{l+s+s2}{\PYZdq{}}
    \PY{k}{else} \PY{n}{d} \PY{o}{;;}
\end{Verbatim}
\fi

\newpage
\section{Abstraction: \texttt{count_people} and the \texttt{List} module.}

Recall the family tree modeling problem from the Lab. (Aside: did you 
like the model? Did you agree with the type definitions?)

Here's a definition of the function \texttt{count_people} from the lab:
% Preliminaries
% 
% definition
% 
\begin{Verbatim}[commandchars=\\\{\}]
\PY{k}{let} \PY{k}{rec} \PY{n}{count\PYZus{}people} \PY{o}{(}\PY{n}{fam} \PY{o}{:} \PY{n}{family}\PY{o}{)} \PY{o}{:} \PY{k+kt}{int} \PY{o}{=}
  \PY{k}{match} \PY{n}{fam} \PY{k}{with}
  \PY{o}{|} \PY{n+nc}{Single} \PY{o}{\PYZus{}} \PY{o}{\PYZhy{}\PYZgt{}} \PY{l+m+mi}{1}
  \PY{o}{|} \PY{n+nc}{Family} \PY{o}{(}\PY{o}{\PYZus{}}\PY{o}{,} \PY{o}{\PYZus{}}\PY{o}{,} \PY{n}{fl}\PY{o}{)} \PY{o}{\PYZhy{}\PYZgt{}} \PY{l+m+mi}{2} \PY{o}{+} \PY{n}{count\PYZus{}people} \PY{n}{fl} \PY{o}{;;}
\end{Verbatim}

Great. Right? (What's the error here?)

How can we improve this implementation? 

\ifsoln
\solution{Here's a version that uses recursion directly. 
This seems a little repetitive, though!}
\begin{Verbatim}[commandchars=\\\{\}]
\PY{o}{\PYZsh{}} \PY{k}{let} \PY{k}{rec} \PY{n}{count\PYZus{}people} \PY{o}{(}\PY{n}{f} \PY{o}{:} \PY{n}{family}\PY{o}{)} \PY{o}{:} \PY{k+kt}{int} \PY{o}{=}
  \PY{c}{(*}\PY{c}{ helper function to sum over family list }\PY{c}{*)} 
  \PY{k}{let} \PY{k}{rec} \PY{n}{count\PYZus{}list} \PY{o}{(}\PY{n}{l} \PY{o}{:} \PY{n}{family} \PY{k+kt}{list}\PY{o}{)} \PY{o}{:} \PY{k+kt}{int} \PY{o}{=}
    \PY{k}{match} \PY{n}{l} \PY{k}{with} \PY{n+nb+bp}{[]} \PY{o}{\PYZhy{}\PYZgt{}} \PY{l+m+mi}{0}
    \PY{o}{|} \PY{n}{hd} \PY{o}{::} \PY{n}{tl} \PY{o}{\PYZhy{}\PYZgt{}} \PY{o}{(}\PY{n}{count\PYZus{}people} \PY{n}{hd}\PY{o}{)} \PY{o}{+} \PY{n}{count\PYZus{}list} \PY{n}{tl} \PY{k}{in}
         
  \PY{k}{match} \PY{n}{f} \PY{k}{with} 
  \PY{n+nc}{Single} \PY{o}{\PYZus{}} \PY{o}{\PYZhy{}\PYZgt{}} \PY{l+m+mi}{1}
  \PY{o}{|} \PY{n+nc}{Family} \PY{o}{(}\PY{o}{\PYZus{}}\PY{o}{,} \PY{o}{\PYZus{}}\PY{o}{,} \PY{n}{fl}\PY{o}{)} \PY{o}{\PYZhy{}\PYZgt{}} \PY{l+m+mi}{2} \PY{o}{+} \PY{n}{count\PYZus{}list} \PY{n}{fl} \PY{o}{;;}
\end{Verbatim}

\solution{I think we can make this even more concise as follows:}

\begin{Verbatim}[commandchars=\\\{\}]
\PY{o}{\PYZsh{}} \PY{k}{let} \PY{k}{rec} \PY{n}{count\PYZus{}people} \PY{o}{(}\PY{n}{fam} \PY{o}{:} \PY{n}{family}\PY{o}{)} \PY{o}{:} \PY{k+kt}{int} \PY{o}{=}
  \PY{k}{match} \PY{n}{fam} \PY{k}{with}
  \PY{o}{|} \PY{n+nc}{Family} \PY{o}{(}\PY{o}{\PYZus{}}\PY{o}{,} \PY{o}{\PYZus{}}\PY{o}{,} \PY{n}{children}\PY{o}{)} \PY{o}{\PYZhy{}\PYZgt{}} 
    \PY{l+m+mi}{2} \PY{o}{+} \PY{n+nn}{List}\PY{p}{.}\PY{n}{fold\PYZus{}left} 
          \PY{o}{(}\PY{k}{fun} \PY{n}{prev} \PY{n}{sub\PYZus{}fam} \PY{o}{\PYZhy{}\PYZgt{}} \PY{n}{count\PYZus{}people} \PY{n}{sub\PYZus{}fam} \PY{o}{+} \PY{n}{prev}\PY{o}{)} 
          \PY{l+m+mi}{0} \PY{n}{children} 
  \PY{o}{|} \PY{o}{\PYZus{}} \PY{o}{\PYZhy{}\PYZgt{}} \PY{l+m+mi}{1}\PY{o}{;;}
\end{Verbatim}

This should be understood as ``collapsing'' the family tree into
a single summary statistic about the whole thing -- the magic is
that we have a fold within a fold (via the recursive call to
\texttt{count_people}) that nests a recursion across the 
list of families \textit{and} down each family in the list
at the same time.
\fi

\newpage
\section{Nested Recursion: \texttt{find_parents}}




\newpage
\section{Problem Set 3: Tips `n` Tricks}

\begin{itemize}

  \item Keep your eyes open for \texttt{fold}ing opportunities. They
        can appear where you least expect them and make for
        very slick code! 

  \item Built in operators are powerful, and OCaml knows
        a lot about how to manipulate values. The \texttt{Pervasives} 
        module has all of the secrets. For example, do you know
        the type of the \texttt{( = )} function? 
        How about other comparators? 
\begin{Verbatim}[commandchars=\\\{\}]
\PY{o}{\PYZsh{}} \PY{o}{(} \PY{o}{=} \PY{o}{)} \PY{o}{;;}
\PY{o}{\PYZhy{}} \PY{o}{:} \PY{k}{\PYZsq{}}\PY{n}{a} \PY{o}{\PYZhy{}\PYZgt{}} \PY{k}{\PYZsq{}}\PY{n}{a} \PY{o}{\PYZhy{}\PYZgt{}} \PY{k+kt}{bool} \PY{o}{=} \PY{o}{\PYZlt{}}\PY{k}{fun}\PY{o}{\PYZgt{}}
\end{Verbatim}
  \item Make conscious choices about what method of record field
        accessing you use. Using the ``easiest'' can make your 
        code much more complicated than necessary!

  \item Be deliberate about understanding the \texttt{times} function
        and the algorithm \textbf{before} you start to code: the
        code will only get in the way.

  \item Appropriately scope helper functions. \textit{If you write a helper function that is only ever called within one other function it should be scoped as such}.

  \item Make sure to read and understand the style guide. 
\end{itemize}



\end{document}














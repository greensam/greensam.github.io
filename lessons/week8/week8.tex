\documentclass[11pt]{amsart}
\usepackage{graphicx}
\usepackage{amssymb}
\usepackage{minted}
\usepackage{etex}

% index
%\usepackage{makeidx}   % not needed in AMS styles
\makeindex
\newcommand{\indx}[1]{\index{#1}#1}
\renewcommand{\seename}{see}    % AMS style has it as "see also"

% fonts
\usepackage{palatino}
\usepackage{pxfonts}
\usepackage{pifont}
\usepackage{alltt}
\usepackage{textcomp}
\usepackage{upquote}

% color 
\usepackage[usenames,dvipsnames]{xcolor}

\newcommand{\highlightcolor}{BrickRed}
\newcommand{\lowlightcolor}{Gray}

% listings, links, etc. 
\usepackage[colorlinks=true,
            urlcolor=\highlightcolor,
            linkcolor=\highlightcolor,
            citecolor=\highlightcolor]{hyperref}
\urlstyle{same}  % don't use monospace font for urls 
\usepackage{listings}

% code listings using minted and pygments
\usepackage{minted}
% At some point, need to write a better pygment style as per http://pygments.org/docs/styles/
\usemintedstyle{bw} %trac
\setminted{xleftmargin=2\parindent}
% This environment definition doesn't work, since minted is a verbatim
% environment that needs to see a real \end{minted} not
% a \end{ocamlex}. Too bad.
%\newenvironment{ocamlex}{\begin{minted}{ocaml}}{\end{minted}}


% Configuration for Caml listings using camltex

\newlength{\abovecamlskip}
\setlength{\abovecamlskip}{\bigskipamount}
\newlength{\belowcamlskip}
\setlength{\belowcamlskip}{\bigskipamount}
\newlength{\camlblanklineskip}
\setlength{\camlblanklineskip}{\medskipamount}

% This command allows putting together multiple caml listings with
% controllable space between them.
\newcommand{\nextcaml}{%
\vspace{\camlblanklineskip}%
}

\newcommand{\ocaml}{\texttt{OCaml}}

% tables
\usepackage{tabulary}
\usepackage{booktabs}
\usepackage[tableposition=below]{caption}
\usepackage{graphicx}

% fancy verbatim environment, useful for typesetting program output
\usepackage{fancyvrb}

% linguistic examples
\usepackage{lingmacros}

% interline spacing
\usepackage{setspace}
\onehalfspacing
%\doublespacing

% em-dashes (some inspiration taken from ltugboat.sty)
\newcommand{\dashpunctskip}{\hspace{.25em}\relax}
%\newcommand{\dd}{\unskip{\,--\,}\ignorespaces}
%\newcommand{\dd}{\unskip{---}\ignorespaces}
%\newcommand{\dd}{\unskip{ -- }\ignorespaces}
\newcommand{\dd}{\unskip\nobreak\dashpunctskip--\dashpunctskip\ignorespaces}

% if using xelatex, the following synonym works
%\newcommand{\—}{\dd}

% a literal (like a filename) which should be presented verbatim, but
% allow line breaks but no hyphens inserted
\newcommand{\literal}[1]{\nolinkurl{#1}}
\usepackage{underscore}
\newcommand{\code}[1]{\texttt{#1}}

% tree drawing
%\usepackage{parsetree}
\usepackage{tikz}
\usepackage{tikz-qtree}

% document structure
\newcommand{\newsection}[1]{\clearpage\section{#1}}
\newcommand{\topic}[1]{\subsection*{#1}}
\setcounter{tocdepth}{1}

% symbols
\usepackage{amssymb}
\usepackage{bbding}
\usepackage{mathtools}
\usepackage{textcomp}
\usepackage{extarrows}  % extensible arrows

% random symbols
\renewcommand{\tilde}{\textasciitilde}

% sets and sequences
\newcommand{\seq}[1]{{\langle #1 \rangle}}
\newcommand{\set}[1]{{\{ #1 \}}}
\newcommand{\vect}[1]{\mathbf{#1}} % variables over strings and sequences
\renewcommand{\emptyset}{\varnothing}
\newcommand{\emptystr}{\varepsilon}
\newcommand{\generic}[1]{\mbox{$\langle$\textrm{\textit{#1}}$\rangle$}}

% floor and ceiling
\newcommand{\floor}[1]{\left\lfloor #1 \right\rfloor}
\newcommand{\ceil}[1]{\left\lceil #1 \right\rceil}

% guillemets
\usepackage[T1]{fontenc}

% operators
\newcommand{\argmax}{\operatornamewithlimits{argmax}}
\newcommand{\cnt}[1]{\sharp(#1)}
\newcommand{\definedby}{\triangleq}
\newcommand{\Prob}{{\Pr}}
\newcommand{\Prcap}{\widehat\Prob}
\DeclareMathOperator{\Perplex}{Pp}
\newcommand{\Pathprob}{{\Prob}_{path}} 
\newcommand{\blt}{\mathop{\bullet}}

% semiring operations and values
% circled plus and times for semirings
%\newcommand{\vplusinternal}{\ooalign{\hss$\circ$\hss\cr$+$}}
%\newcommand{\vtimesinternal}{\ooalign{\hss$\circ$\hss\cr$\times$}}
\newcommand{\vplus}{\mathrel{\mathchoice{\vplusinternal}{\vplusinternal}{\scriptsize\vplusinternal}{\tiny\vplusinternal}}}
\newcommand{\vtimes}{\mathrel{\mathchoice{\vtimesinternal}{\vtimesinternal}{\scriptsize\vtimesinternal}{\tiny\vtimesinternal}}}
%\newcommand{\vplus}{\oplus}
%\newcommand{\vtimes}{\otimes}
%\newcommand{\vzero}{{\ooalign{\hss$\circ$\hss\cr$0$}}}
%\newcommand{\vone}{{\ooalign{\hss$\circ$\hss\cr$
%\usepackage{fge}
%\newcommand{\vzero}{\fgestruckzero}
%\newcommand{\vone}{\fgestruckone}
%\newcommand{\vzero}{0}
%\newcommand{\vone}{1}

% numerals
\newcommand{\num}[1]{\underline{\ensuremath{#1}}}
\newcommand{\vzero}{\num{0}}
\newcommand{\vone}{\num{1}}
\newcommand{\vplusinternal}{\underline{+}}
\newcommand{\vtimesinternal}{\underline{\times}}

% logic
\newcommand{\Land}{\wedge}
\newcommand{\Lor}{\vee}
\newcommand{\Lnot}{\neg}
\newcommand{\Lif}{\rightharpoonup}
\newcommand{\Liff}{\rightleftharpoons}
\newcommand{\Ltrue}{T}
\newcommand{\Lfalse}{F}

% types
\newcommand{\cross}{\times}
%\newcommand{\to}{\rightarrow}

% lambda calculus
\newcommand{\subst}[3]{#1[#3 \mapsto #2]}   % subst P for x in Q is \subst{Q}{P}{x}
\newcommand{\rewrites}[1]{\xlongrightarrow{\ #1\ }}
\newcommand{\rwrts}{\rewrites{}}
\newcommand{\redex}[1]{\underline{#1}}

% OCaml rewriting
\newcommand{\oredex}[1]{\underline{#1}}
\newcommand{\orwrts}{\(\rwrts\)}

% better spacing in conditional probabilities, instead of
% \newcommand{\given}{|}           % too little space
%\newcommand{\given}{\mid}     % too much space
\newcommand{\given}{\,|\,}      % just right

% values of random variables
\newcommand{\rvval}[1]{\mathsf{#1}}

% indicator functions
\newcommand{\indic}[1]{\mathbb{I}\left[#1\right]}

% start and end of string symbols
\newcommand{\strbeg}{\mbox{\guillemotleft}}
\newcommand{\strend}{\mbox{\guillemotright}}

% different types of symbols
\newcommand{\sym}[1]{\mbox{#1}}
% nonterminals and terminals
\newcommand{\ntm}[1]{\mbox{\textit{#1}}}
\newcommand{\tm}[1]{\mbox{\textrm{#1}}}
\newcommand{\term}[1]{\ntm{#1}} % deprecated; use \nt
\newcommand{\func}[1]{\operatorname{\mathit{#1}}}

% grammar rules
\newcommand{\goesto}{\rightarrow}

% punctuation in equations
\newcommand{\eqpunc}[1]{{\makebox[0pt][l]{\qquad\rm{#1}}}}

% first mentions and other marginalia
\newcommand{\firstusefont}[1]{\textsc{#1}}
\newcommand{\firstuse}[1]{\firstusemarked{#1}{#1}}
% first use with separate margin and inline (= index) forms
\newcommand{\firstusemarked}[2]{\firstusemarkedindexed{#1}{#2}{#1}}
% first use with separate margin, inline, and index forms
%   for placing the index item at a separate location in the index, use 
%   \firstusemarked{margin}{inline}{index@indexloc}, e.g.,
%   \firstusemarked{$n$-gram}{$n$-gram}{n-gram@$n$-gram}
\newcommand{\firstusemarkedindexed}[3]{\index{#3}\firstusefont{#2}%
  \marginpar[\begin{spacing}{.75}\vspace{-.75ex}\raggedleft\scriptsize\firstusefont{\raggedleft\color{\highlightcolor} #1}\end{spacing}\vspace*{2.75ex}]{\begin{spacing}{.75}\vspace{-.75ex}\raggedright\scriptsize\firstusefont{\raggedright\color{\highlightcolor} #1}\end{spacing}\vspace*{2.75ex}}}


% instructor material
\newif\ifinstructor
\instructortrue
\instructorfalse

\newcommand{\instructor}[1]{\ifinstructor\begin{instructorchunk}#1\end{instructorchunk}\fi}
\newenvironment{instructorchunk}%
{\marginpar[\raggedleft\color{red}\PencilRight]{\raggedright\color{red}\PencilLeft}\color{red}}{}

\newcommand{\inclass}[2]{\ifinstructor\begin{instructorchunk}\begin{description}\item[Question]#1\par\item[Answer]#2\end{description}\end{instructorchunk}\fi}


% displaying shell commands and interactive python

\DefineVerbatimEnvironment%
{commandblock}{Verbatim}
{commandchars=\\\{\}, xleftmargin=2\parindent}

\newcommand{\command}[1]{%
\begin{quote}
\begin{alltt}
#1
\end{alltt}
\end{quote}}
\newcommand{\shellcommand}[1]{\command{\$ #1}}

% advanced topics

\usepackage{manfnt}
%\newcommand{\advanced}[1]{\footnote{\manimpossiblecube: #1}}
%\newcommand{\advanced}[1]{\begin{quote}\color{\lowlightcolor}\manimpossiblecube~\small
%    #1\end{quote}}

%\newenvironment{advanced}{\begin{quote}\color{\lowlightcolor}\manimpossiblecube~\small}{\end{quote}}

\newenvironment{advanced}{\begin{list}{}{%
      \setlength{\leftmargin}{15ex}%
      \setlength{\rightmargin}{0ex}%
    }
  \item \color{\highlightcolor}\manimpossiblecube\hspace{2ex}\small \color{\lowlightcolor} }{\end{list}}

%\renewenvironment{advanced}{\par\begingroup\color{gray}\makebox[0pt][r]{\manimpossiblecube\hspace{5ex}}\small}{\endgroup\par}

% bibliography
\usepackage[colon]{natbib}

% theorems
\newtheorem{theorem}{Theorem}

% problems and exercises
\newcommand{\probmark}{\hfill{$\Box$}}
\newtheorem{probleminternal}{Problem}
\newtheorem{questioninternal}{Question}
\newtheorem{exerciseinternal}[probleminternal]{Exercise}% 
\newtheorem{stageinternal}[probleminternal]{Stage}% 
\newenvironment{exercise}{\begin{exerciseinternal}}{\probmark\end{exerciseinternal}}
\newenvironment{problem}{\begin{probleminternal}}{\probmark\end{probleminternal}}
\newenvironment{stage}{\begin{stageinternal}}{\probmark\end{stageinternal}}
\newenvironment{question}{\begin{questioninternal}}{\probmark\end{questioninternal}}
\newtheorem{definition}{Definition}

\newcommand{\nextproblem}{\vspace{2em}}

\newcommand{\solution}[1]{{\color{\highlightcolor}
   \begin{quote}\textbf{Solution:} #1\end{quote}
   }}
 
% temporary notes
\newcommand{\note}[1]{{\color{red}{[[[#1]]]}}}
%\newcommand{\note}[1]{{\color{red} \textbf{[[[#1]]]}}}
\renewcommand{\note}[1]{}

\sloppy

% line numbering
% Don't need lineno since minted loads it.
%\usepackage[switch*,mathlines]{lineno}
\runningpagewiselinenumbers
\setpagewiselinenumbers
\switchlinenumbers*
\linenumberdisplaymath
\modulolinenumbers[5]
\renewcommand\linenumberfont{\normalfont\tiny\sffamily\color{Tan}}

\usepackage{bussproofs}
\newcommand{\SideCond}[1]{\RightLabel{\ #1}}

\newcommand{\types}{\mathcal{T}}
\newcommand{\consts}[1]{\mathcal{C}_{#1}}
\newcommand{\vars}[1]{\mathcal{X}_{#1}}
\newcommand{\fto}[2]{\langle #1, #2 \rangle}
\newcommand{\domain}[2]{\mathcal{D}_{#1,#2}}
\newcommand{\true}{\mathsf{True}}
\newcommand{\false}{\mathsf{False}}
\newcommand{\model}{\mathcal{M}}
\newcommand{\assign}{\mathcal{G}}
\newcommand{\dennoparams}[1]{\llbracket #1 \rrbracket}
\newcommand{\den}[3]{\dennoparams{#1}_{#2,#3}}
\newcommand{\denmg}[1]{\den{#1}{\model}{\assign}}



\title{Event Programming and Next Steps}
\author{Sam Green}
\date{\today}                               



\usepackage{fancyvrb,color}

\makeatletter
\def\PY@reset{\let\PY@it=\relax \let\PY@bf=\relax%
    \let\PY@ul=\relax \let\PY@tc=\relax%
    \let\PY@bc=\relax \let\PY@ff=\relax}
\def\PY@tok#1{\csname PY@tok@#1\endcsname}
\def\PY@toks#1+{\ifx\relax#1\empty\else%
    \PY@tok{#1}\expandafter\PY@toks\fi}
\def\PY@do#1{\PY@bc{\PY@tc{\PY@ul{%
    \PY@it{\PY@bf{\PY@ff{#1}}}}}}}
\def\PY#1#2{\PY@reset\PY@toks#1+\relax+\PY@do{#2}}

\expandafter\def\csname PY@tok@ch\endcsname{\let\PY@it=\textit\def\PY@tc##1{\textcolor[rgb]{1.00,0.00,0.00}{##1}}}
\expandafter\def\csname PY@tok@cm\endcsname{\let\PY@it=\textit\def\PY@tc##1{\textcolor[rgb]{1.00,0.00,0.00}{##1}}}
\expandafter\def\csname PY@tok@cs\endcsname{\let\PY@it=\textit\def\PY@tc##1{\textcolor[rgb]{1.00,0.00,0.00}{##1}}}
\expandafter\def\csname PY@tok@cp\endcsname{\let\PY@it=\textit\def\PY@tc##1{\textcolor[rgb]{1.00,0.00,0.00}{##1}}}
\expandafter\def\csname PY@tok@s2\endcsname{\def\PY@tc##1{\textcolor[rgb]{0.00,0.61,0.00}{##1}}}
\expandafter\def\csname PY@tok@s1\endcsname{\def\PY@tc##1{\textcolor[rgb]{0.00,0.61,0.00}{##1}}}
\expandafter\def\csname PY@tok@nc\endcsname{\def\PY@tc##1{\textcolor[rgb]{0.00,0.46,0.46}{##1}}}
\expandafter\def\csname PY@tok@nd\endcsname{\def\PY@tc##1{\textcolor[rgb]{0.80,0.00,0.64}{##1}}}
\expandafter\def\csname PY@tok@si\endcsname{\def\PY@tc##1{\textcolor[rgb]{0.00,0.61,0.00}{##1}}}
\expandafter\def\csname PY@tok@nf\endcsname{\def\PY@tc##1{\textcolor[rgb]{0.76,0.31,0.00}{##1}}}
\expandafter\def\csname PY@tok@sh\endcsname{\def\PY@tc##1{\textcolor[rgb]{0.00,0.61,0.00}{##1}}}
\expandafter\def\csname PY@tok@c1\endcsname{\let\PY@it=\textit\def\PY@tc##1{\textcolor[rgb]{1.00,0.00,0.00}{##1}}}
\expandafter\def\csname PY@tok@kc\endcsname{\def\PY@tc##1{\textcolor[rgb]{0.00,0.00,1.00}{##1}}}
\expandafter\def\csname PY@tok@c\endcsname{\let\PY@it=\textit\def\PY@tc##1{\textcolor[rgb]{1.00,0.00,0.00}{##1}}}
\expandafter\def\csname PY@tok@sx\endcsname{\def\PY@tc##1{\textcolor[rgb]{0.00,0.61,0.00}{##1}}}
\expandafter\def\csname PY@tok@kd\endcsname{\def\PY@tc##1{\textcolor[rgb]{0.00,0.00,1.00}{##1}}}
\expandafter\def\csname PY@tok@ss\endcsname{\def\PY@tc##1{\textcolor[rgb]{0.00,0.61,0.00}{##1}}}
\expandafter\def\csname PY@tok@sr\endcsname{\def\PY@tc##1{\textcolor[rgb]{0.00,0.61,0.00}{##1}}}
\expandafter\def\csname PY@tok@k\endcsname{\def\PY@tc##1{\textcolor[rgb]{0.00,0.00,1.00}{##1}}}
\expandafter\def\csname PY@tok@kn\endcsname{\def\PY@tc##1{\textcolor[rgb]{0.00,0.00,1.00}{##1}}}
\expandafter\def\csname PY@tok@cpf\endcsname{\let\PY@it=\textit\def\PY@tc##1{\textcolor[rgb]{1.00,0.00,0.00}{##1}}}
\expandafter\def\csname PY@tok@kr\endcsname{\def\PY@tc##1{\textcolor[rgb]{0.00,0.00,1.00}{##1}}}
\expandafter\def\csname PY@tok@s\endcsname{\def\PY@tc##1{\textcolor[rgb]{0.00,0.61,0.00}{##1}}}
\expandafter\def\csname PY@tok@kp\endcsname{\def\PY@tc##1{\textcolor[rgb]{0.00,0.00,1.00}{##1}}}
\expandafter\def\csname PY@tok@kt\endcsname{\def\PY@tc##1{\textcolor[rgb]{0.00,0.00,1.00}{##1}}}
\expandafter\def\csname PY@tok@sc\endcsname{\def\PY@tc##1{\textcolor[rgb]{0.00,0.61,0.00}{##1}}}
\expandafter\def\csname PY@tok@sb\endcsname{\def\PY@tc##1{\textcolor[rgb]{0.00,0.61,0.00}{##1}}}
\expandafter\def\csname PY@tok@se\endcsname{\def\PY@tc##1{\textcolor[rgb]{0.00,0.61,0.00}{##1}}}
\expandafter\def\csname PY@tok@sd\endcsname{\def\PY@tc##1{\textcolor[rgb]{0.00,0.61,0.00}{##1}}}

\def\PYZbs{\char`\\}
\def\PYZus{\char`\_}
\def\PYZob{\char`\{}
\def\PYZcb{\char`\}}
\def\PYZca{\char`\^}
\def\PYZam{\char`\&}
\def\PYZlt{\char`\<}
\def\PYZgt{\char`\>}
\def\PYZsh{\char`\#}
\def\PYZpc{\char`\%}
\def\PYZdl{\char`\$}
\def\PYZhy{\char`\-}
\def\PYZsq{\char`\'}
\def\PYZdq{\char`\"}
\def\PYZti{\char`\~}
% for compatibility with earlier versions
\def\PYZat{@}
\def\PYZlb{[}
\def\PYZrb{]}
\makeatother
\begin{document}
\maketitle

\section{API Design}

Yesterday's lab focused on an implementation
of the event system. To this point, we
haven't talked much about what makes
a useful API or a good interface. Let's 
spend some time examining this interface.

\begin{Verbatim}[commandchars=\\\{\}]
\PY{k}{module} \PY{k}{type} \PY{n+nc}{WEVENT} \PY{o}{=}
\PY{k}{sig}
  \PY{c}{(*}\PY{c}{ The event listener identifier type. }\PY{c}{*)}
  \PY{k}{type} \PY{n}{id}

  \PY{c}{(*}\PY{c}{ The event type. }\PY{c}{*)}
  \PY{k}{type} \PY{k}{\PYZsq{}}\PY{n}{a} \PY{n}{event}

  \PY{c}{(*}\PY{c}{ Create a new event. }\PY{c}{*)}
  \PY{k}{val} \PY{n}{new\PYZus{}event} \PY{o}{:} \PY{k+kt}{unit} \PY{o}{\PYZhy{}\PYZgt{}} \PY{k}{\PYZsq{}}\PY{n}{a} \PY{n}{event}

  \PY{c}{(*}\PY{c}{ Add a listener to an event which is called every time the event}
\PY{c}{     is fired. Return an identifier for the listener. }\PY{c}{*)}
  \PY{k}{val} \PY{n}{add\PYZus{}listener} \PY{o}{:} \PY{k}{\PYZsq{}}\PY{n}{a} \PY{n}{event} \PY{o}{\PYZhy{}\PYZgt{}} \PY{o}{(}\PY{k}{\PYZsq{}}\PY{n}{a} \PY{o}{\PYZhy{}\PYZgt{}} \PY{k+kt}{unit}\PY{o}{)} \PY{o}{\PYZhy{}\PYZgt{}} \PY{n}{id}

  \PY{c}{(*}\PY{c}{ Remove a listener from being called when an event is fired. Has no effect}
\PY{c}{     if the listener is not waiting for the event. }\PY{c}{*)}
  \PY{k}{val} \PY{n}{remove\PYZus{}listener} \PY{o}{:} \PY{k}{\PYZsq{}}\PY{n}{a} \PY{n}{event} \PY{o}{\PYZhy{}\PYZgt{}} \PY{n}{id} \PY{o}{\PYZhy{}\PYZgt{}} \PY{k+kt}{unit}

  \PY{c}{(*}\PY{c}{ Signal that an event has occurred.  The \PYZsq{}a value is passed to each}
\PY{c}{     function waiting for the event. }\PY{c}{*)}
  \PY{k}{val} \PY{n}{fire\PYZus{}event} \PY{o}{:} \PY{k}{\PYZsq{}}\PY{n}{a} \PY{n}{event} \PY{o}{\PYZhy{}\PYZgt{}} \PY{k}{\PYZsq{}}\PY{n}{a} \PY{o}{\PYZhy{}\PYZgt{}} \PY{k+kt}{unit}
\PY{k}{end}
\end{Verbatim}

Questions to consider:
\begin{itemize}
  \item Is the API well-documented?
  \item Are the functions well-named?
  \item Does the API provide all of the
    necessary tools?
    \ifsoln
    \solution{} In general, I would
    say yes, but it's offloads some 
    management onto the client. Think
    of what happened when we had to 
    change all of the listeners for 
    the \texttt{newswire} event, 
    for example.
    \fi


  \item What, if anything, is missing?
  \ifsoln
  \solution{Some candidates:}
  \begin{itemize}
    \item One-shot listener function.
    \item Get callbacks by id function.
    \item Delete all callbacks / reset event
    method. 
    \item Fire multiple times function.
  \end{itemize}
  \fi
\end{itemize}

Is this module
adequate for a full event system? 

\section{Implementation Questions}

Let's look in sequence at each of the
portions of the implementation of the
WEvent module and see they could be
improved or how they are bad.

\subsection{Event and waiter implementation}

We defined the following type to represent
events. What do you think of them?

\begin{Verbatim}[commandchars=\\\{\}]
  \PY{k}{type} \PY{k}{\PYZsq{}}\PY{n}{a} \PY{n}{waiter} \PY{o}{=} \PY{o}{\PYZob{}}\PY{n}{id} \PY{o}{:} \PY{n}{id} \PY{o}{;} \PY{n}{action} \PY{o}{:} \PY{k}{\PYZsq{}}\PY{n}{a} \PY{o}{\PYZhy{}\PYZgt{}} \PY{k+kt}{unit}\PY{o}{\PYZcb{}}
  \PY{k}{type} \PY{k}{\PYZsq{}}\PY{n}{a} \PY{n}{event} \PY{o}{=} \PY{k}{\PYZsq{}}\PY{n}{a} \PY{n}{waiter} \PY{k+kt}{list} \PY{n}{ref} \PY{o}{;;}
\end{Verbatim}

What do you think of this implementation?
Are there any problems with scalability
here? Readability? Maintainability? 

\ifsoln
Some things to think about:
\begin{itemize}
  \item The system is set up around random
  access. Many event systems require
  registering lots of events, and graphics
  systems also in general need to be
  performant. What are the asymptotic times
  for lookup and delete for callbacks? What
  simple data structure could we use instead?
  \item Why do we need a \texttt{ref}
        event type? Why aren't waiters
        exposed to the client? 
\end{itemize}
\fi

\section{Fire event implementation}

What do you think of this implementation
of the fire_event function?

\begin{Verbatim}[commandchars=\\\{\}]
  \PY{k}{let} \PY{n}{fire\PYZus{}event} \PY{o}{(}\PY{n}{e} \PY{o}{:} \PY{k}{\PYZsq{}}\PY{n}{a} \PY{n}{event}\PY{o}{)} \PY{o}{(}\PY{n}{v} \PY{o}{:} \PY{k}{\PYZsq{}}\PY{n}{a}\PY{o}{)} \PY{o}{:} \PY{k+kt}{unit} \PY{o}{=}
    \PY{k}{let} \PY{n}{waiters} \PY{o}{=} \PY{o}{!}\PY{n}{e} \PY{k}{in}
    \PY{k}{let} \PY{o}{\PYZus{}} \PY{o}{=} \PY{n+nn}{List}\PY{p}{.}\PY{n}{map} \PY{o}{(}\PY{k}{fun} \PY{n}{w} \PY{o}{\PYZhy{}\PYZgt{}} \PY{n}{w}\PY{o}{.}\PY{n}{action} \PY{n}{v}\PY{o}{)}
    \PY{n}{waiters} \PY{k}{in} \PY{n+nb+bp}{()}
\end{Verbatim}

\section{Error-Message Interpretation}
(Thanks to Katherine Binney for this
exercise).

Many people in the lab yesterday started
with following as their implementation
of events. 
\begin{Verbatim}[commandchars=\\\{\}]
\PY{k}{let} \PY{n}{newswire} \PY{o}{=} \PY{k}{fun} \PY{o}{\PYZus{}} \PY{o}{\PYZhy{}\PYZgt{}} \PY{n+nn}{WEvent}\PY{p}{.}\PY{n}{new\PYZus{}event}\PY{o}{;}
\end{Verbatim}
but this resulted in the following error
message later on: \\

\begin{Verbatim}[commandchars=\\\{\}]
\PY{k}{let} \PY{n}{fnn\PYZus{}listener} \PY{o}{=}
    \PY{n+nn}{WEvent}\PY{p}{.}\PY{n}{add\PYZus{}listener} \PY{n}{newswire} \PY{n}{fakeNewsNetwork} \PY{o}{;;}
\PY{n+nc}{Error}\PY{o}{:} \PY{n+nc}{This} \PY{n}{expression} \PY{n}{has} \PY{k}{type} \PY{k}{\PYZsq{}}\PY{n}{a} \PY{o}{\PYZhy{}\PYZgt{}} \PY{k+kt}{unit} \PY{o}{\PYZhy{}\PYZgt{}} \PY{k}{\PYZsq{}}\PY{n}{b} \PY{n+nn}{WEvent}\PY{p}{.}\PY{n}{event}                                               \PY{n}{but} \PY{n}{an} \PY{n}{expression} \PY{n}{was} \PY{n}{expected} \PY{k}{of} \PY{k}{type} \PY{k}{\PYZsq{}}\PY{n}{c} \PY{n+nn}{WEvent}\PY{p}{.}\PY{n}{event}
\end{Verbatim}

As practice for the midterm and the final
project, what
does this error mean? How do we debug it? 
\\

(Hint: I've said it a few times this
semester.)

\section{Type Annotation Quick Hits}

There were a few interesting examples
of ``delayed'' type inference this week.
To see what I mean, let's do the following
exercises (thanks to Katherine Binney,
again):

\begin{exercise} Write an OCaml expression
for an empty string list.
\ifsoln:
Here's an empty list:
\begin{Verbatim}[commandchars=\\\{\}]
\PY{o}{\PYZsh{}} \PY{k}{let} \PY{n}{x} \PY{o}{=} \PY{n+nb+bp}{[]} \PY{o}{;;}
\PY{k}{val} \PY{n}{x} \PY{o}{:} \PY{k}{\PYZsq{}}\PY{n}{a} \PY{k+kt}{list} \PY{o}{=} \PY{n+nb+bp}{[]}
\end{Verbatim}
But it's polymorphic! Remember,
we can explicitly type it on 2 ways.
\begin{Verbatim}[commandchars=\\\{\}]
\PY{o}{\PYZsh{}} \PY{k}{let} \PY{n}{x} \PY{o}{:} \PY{k+kt}{string} \PY{k+kt}{list} \PY{o}{=} \PY{n+nb+bp}{[]} \PY{o}{;;}
\PY{k}{val} \PY{n}{x} \PY{o}{:} \PY{k+kt}{string} \PY{k+kt}{list} \PY{o}{=} \PY{n+nb+bp}{[]}
\end{Verbatim}
or 
\begin{Verbatim}[commandchars=\\\{\}]
\PY{o}{\PYZsh{}} \PY{k}{let} \PY{n}{x} \PY{o}{=} \PY{n+nb+bp}{[]} \PY{o}{;;}
\PY{k}{val} \PY{n}{x} \PY{o}{:} \PY{k}{\PYZsq{}}\PY{n}{a} \PY{k+kt}{list} \PY{o}{=} \PY{n+nb+bp}{[]}
\PY{o}{\PYZsh{}} \PY{l+s+s2}{\PYZdq{}}\PY{l+s+s2}{hello}\PY{l+s+s2}{\PYZdq{}}\PY{o}{::}\PY{n}{x} \PY{o}{;;}
\PY{o}{\PYZhy{}} \PY{o}{:} \PY{k+kt}{string} \PY{k+kt}{list} \PY{o}{=} \PY{o}{[}\PY{l+s+s2}{\PYZdq{}}\PY{l+s+s2}{hello}\PY{l+s+s2}{\PYZdq{}}\PY{o}{]}
\PY{o}{\PYZsh{}} \PY{n}{x} \PY{o}{;;}
\PY{o}{\PYZhy{}} \PY{o}{:} \PY{k}{\PYZsq{}}\PY{n}{a} \PY{k+kt}{list} \PY{o}{=} \PY{n+nb+bp}{[]}
\end{Verbatim}
Wait, what?? Shouldn't this last \texttt{x}
have type-checked as a string list? No! 
The type of the value \texttt{x} hasn't 
actually changed. How does this 
compare to the way polymorphic events
type-checked?
\fi
\end{exercise}

\begin{exercise} Write a function
\texttt{map_int} that maps over lists
of integers. 
\ifsoln
\begin{Verbatim}[commandchars=\\\{\}]
\PY{o}{\PYZsh{}} \PY{k}{let} \PY{n}{map\PYZus{}int} \PY{o}{(}\PY{n}{f} \PY{o}{:} \PY{k+kt}{int} \PY{o}{\PYZhy{}\PYZgt{}} \PY{k}{\PYZsq{}}\PY{n}{b}\PY{o}{)} \PY{o}{(}\PY{n}{lst} \PY{o}{:} \PY{k+kt}{int} \PY{k+kt}{list}\PY{o}{)} \PY{o}{:} \PY{k}{\PYZsq{}}\PY{n}{b} \PY{k+kt}{list} \PY{o}{=} 
  \PY{k}{let} \PY{k}{rec} \PY{n}{map\PYZsq{}} \PY{n}{lst} \PY{o}{=} 
    \PY{k}{match} \PY{n}{lst} \PY{k}{with} 
    \PY{o}{|} \PY{n+nb+bp}{[]} \PY{o}{\PYZhy{}\PYZgt{}} \PY{n+nb+bp}{[]}
    \PY{o}{|} \PY{n}{h} \PY{o}{::} \PY{n}{t} \PY{o}{\PYZhy{}\PYZgt{}} \PY{o}{(}\PY{n}{f} \PY{n}{h}\PY{o}{)} \PY{o}{::} \PY{o}{(}\PY{n}{map\PYZsq{}} \PY{n}{t}\PY{o}{)} \PY{k}{in} 
  \PY{n}{map\PYZsq{}} \PY{n}{lst}
\PY{o}{;;}
\PY{k}{val} \PY{n}{map\PYZus{}int} \PY{o}{:} \PY{o}{(}\PY{k+kt}{int} \PY{o}{\PYZhy{}\PYZgt{}} \PY{k}{\PYZsq{}}\PY{n}{b}\PY{o}{)} \PY{o}{\PYZhy{}\PYZgt{}} \PY{k+kt}{int} \PY{k+kt}{list} \PY{o}{\PYZhy{}\PYZgt{}} \PY{k}{\PYZsq{}}\PY{n}{b} \PY{k+kt}{list} \PY{o}{=} \PY{o}{\PYZlt{}}\PY{k}{fun}\PY{o}{\PYZgt{}}
\end{Verbatim}
\fi
\end{exercise}

\section{Understanding error messages (ie be the TF)}
Note: Taken from (mostly private) piazza
questions. We think giving you fish is ok,
but we also want to teach you to fish.
(Thanks to Katherine Binney, again.)

\begin{exercise} What's the likely
cause of this error message?
\begin{Verbatim}[commandchars=\\\{\}]
\PY{k}{let} \PY{n}{m1} \PY{o}{=} \PY{k}{new} \PY{n}{mass} \PY{l+m+mi}{0}\PY{o}{.} \PY{l+m+mi}{0}\PY{o}{.} \PY{o}{;;}
\PY{k}{let} \PY{o}{\PYZus{}} \PY{o}{=} \PY{n}{m1}\PY{o}{\PYZsh{}}\PY{n}{move}  \PY{n}{m1}\PY{o}{;;}

\PY{l+s+s2}{\PYZdq{}}\PY{l+s+s2}{Error: This expression has type float \PYZhy{}\PYZgt{} Masses.mass}
\PY{l+s+s2}{       It has no method move}\PY{l+s+s2}{\PYZdq{}}
\end{Verbatim}
\end{exercise}

\begin{exercise} This error message is
a bit trickier. 
\begin{Verbatim}[commandchars=\\\{\}]
\PY{k}{let} \PY{n}{g} \PY{o}{=} \PY{k}{new} \PY{n}{point} \PY{o}{\PYZhy{}}\PY{l+m+mi}{1}\PY{o}{.}\PY{l+m+mi}{0} \PY{l+m+mi}{0}\PY{o}{.}\PY{l+m+mi}{0}\PY{o}{;;}
\PY{l+s+s2}{\PYZdq{}}\PY{l+s+s2}{Error: This expression has type float \PYZhy{}\PYZgt{} float \PYZhy{}\PYZgt{} point }
\PY{l+s+s2}{but an expression was expected of type int.}\PY{l+s+s2}{\PYZdq{}}
\end{Verbatim}
\end{exercise}

\section{Next Steps}

Hopefully CS 51 has you excited about
future courses in Computer Science. 

Courses I recommend: 

\begin{itemize}
  \item CS 136
  \item CS 121
  \item CS 124
  \item CS 61 
  \item CS 134 
  \item Stat 110
  \item CS 187 (I hear the usual Prof. is
  GREAT.)
\end{itemize}

I am happy to talk about them and others! 
I also encourage you to be brave about
200-level courses -- I wasn't particularly,
and I missed out on some good stuff. 

\end{document}  
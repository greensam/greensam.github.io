\documentclass[12p]{amsart}

\usepackage{etex}

% index
%\usepackage{makeidx}   % not needed in AMS styles
\makeindex
\newcommand{\indx}[1]{\index{#1}#1}
\renewcommand{\seename}{see}    % AMS style has it as "see also"

% fonts
\usepackage{palatino}
\usepackage{pxfonts}
\usepackage{pifont}
\usepackage{alltt}
\usepackage{textcomp}
\usepackage{upquote}

% color 
\usepackage[usenames,dvipsnames]{xcolor}

\newcommand{\highlightcolor}{BrickRed}
\newcommand{\lowlightcolor}{Gray}

% listings, links, etc. 
\usepackage[colorlinks=true,
            urlcolor=\highlightcolor,
            linkcolor=\highlightcolor,
            citecolor=\highlightcolor]{hyperref}
\urlstyle{same}  % don't use monospace font for urls 
\usepackage{listings}

% code listings using minted and pygments
\usepackage{minted}
% At some point, need to write a better pygment style as per http://pygments.org/docs/styles/
\usemintedstyle{bw} %trac
\setminted{xleftmargin=2\parindent}
% This environment definition doesn't work, since minted is a verbatim
% environment that needs to see a real \end{minted} not
% a \end{ocamlex}. Too bad.
%\newenvironment{ocamlex}{\begin{minted}{ocaml}}{\end{minted}}


% Configuration for Caml listings using camltex

\newlength{\abovecamlskip}
\setlength{\abovecamlskip}{\bigskipamount}
\newlength{\belowcamlskip}
\setlength{\belowcamlskip}{\bigskipamount}
\newlength{\camlblanklineskip}
\setlength{\camlblanklineskip}{\medskipamount}

% This command allows putting together multiple caml listings with
% controllable space between them.
\newcommand{\nextcaml}{%
\vspace{\camlblanklineskip}%
}

\newcommand{\ocaml}{\texttt{OCaml}}

% tables
\usepackage{tabulary}
\usepackage{booktabs}
\usepackage[tableposition=below]{caption}
\usepackage{graphicx}

% fancy verbatim environment, useful for typesetting program output
\usepackage{fancyvrb}

% linguistic examples
\usepackage{lingmacros}

% interline spacing
\usepackage{setspace}
\onehalfspacing
%\doublespacing

% em-dashes (some inspiration taken from ltugboat.sty)
\newcommand{\dashpunctskip}{\hspace{.25em}\relax}
%\newcommand{\dd}{\unskip{\,--\,}\ignorespaces}
%\newcommand{\dd}{\unskip{---}\ignorespaces}
%\newcommand{\dd}{\unskip{ -- }\ignorespaces}
\newcommand{\dd}{\unskip\nobreak\dashpunctskip--\dashpunctskip\ignorespaces}

% if using xelatex, the following synonym works
%\newcommand{\—}{\dd}

% a literal (like a filename) which should be presented verbatim, but
% allow line breaks but no hyphens inserted
\newcommand{\literal}[1]{\nolinkurl{#1}}
\usepackage{underscore}
\newcommand{\code}[1]{\texttt{#1}}

% tree drawing
%\usepackage{parsetree}
\usepackage{tikz}
\usepackage{tikz-qtree}

% document structure
\newcommand{\newsection}[1]{\clearpage\section{#1}}
\newcommand{\topic}[1]{\subsection*{#1}}
\setcounter{tocdepth}{1}

% symbols
\usepackage{amssymb}
\usepackage{bbding}
\usepackage{mathtools}
\usepackage{textcomp}
\usepackage{extarrows}  % extensible arrows

% random symbols
\renewcommand{\tilde}{\textasciitilde}

% sets and sequences
\newcommand{\seq}[1]{{\langle #1 \rangle}}
\newcommand{\set}[1]{{\{ #1 \}}}
\newcommand{\vect}[1]{\mathbf{#1}} % variables over strings and sequences
\renewcommand{\emptyset}{\varnothing}
\newcommand{\emptystr}{\varepsilon}
\newcommand{\generic}[1]{\mbox{$\langle$\textrm{\textit{#1}}$\rangle$}}

% floor and ceiling
\newcommand{\floor}[1]{\left\lfloor #1 \right\rfloor}
\newcommand{\ceil}[1]{\left\lceil #1 \right\rceil}

% guillemets
\usepackage[T1]{fontenc}

% operators
\newcommand{\argmax}{\operatornamewithlimits{argmax}}
\newcommand{\cnt}[1]{\sharp(#1)}
\newcommand{\definedby}{\triangleq}
\newcommand{\Prob}{{\Pr}}
\newcommand{\Prcap}{\widehat\Prob}
\DeclareMathOperator{\Perplex}{Pp}
\newcommand{\Pathprob}{{\Prob}_{path}} 
\newcommand{\blt}{\mathop{\bullet}}

% semiring operations and values
% circled plus and times for semirings
%\newcommand{\vplusinternal}{\ooalign{\hss$\circ$\hss\cr$+$}}
%\newcommand{\vtimesinternal}{\ooalign{\hss$\circ$\hss\cr$\times$}}
\newcommand{\vplus}{\mathrel{\mathchoice{\vplusinternal}{\vplusinternal}{\scriptsize\vplusinternal}{\tiny\vplusinternal}}}
\newcommand{\vtimes}{\mathrel{\mathchoice{\vtimesinternal}{\vtimesinternal}{\scriptsize\vtimesinternal}{\tiny\vtimesinternal}}}
%\newcommand{\vplus}{\oplus}
%\newcommand{\vtimes}{\otimes}
%\newcommand{\vzero}{{\ooalign{\hss$\circ$\hss\cr$0$}}}
%\newcommand{\vone}{{\ooalign{\hss$\circ$\hss\cr$
%\usepackage{fge}
%\newcommand{\vzero}{\fgestruckzero}
%\newcommand{\vone}{\fgestruckone}
%\newcommand{\vzero}{0}
%\newcommand{\vone}{1}

% numerals
\newcommand{\num}[1]{\underline{\ensuremath{#1}}}
\newcommand{\vzero}{\num{0}}
\newcommand{\vone}{\num{1}}
\newcommand{\vplusinternal}{\underline{+}}
\newcommand{\vtimesinternal}{\underline{\times}}

% logic
\newcommand{\Land}{\wedge}
\newcommand{\Lor}{\vee}
\newcommand{\Lnot}{\neg}
\newcommand{\Lif}{\rightharpoonup}
\newcommand{\Liff}{\rightleftharpoons}
\newcommand{\Ltrue}{T}
\newcommand{\Lfalse}{F}

% types
\newcommand{\cross}{\times}
%\newcommand{\to}{\rightarrow}

% lambda calculus
\newcommand{\subst}[3]{#1[#3 \mapsto #2]}   % subst P for x in Q is \subst{Q}{P}{x}
\newcommand{\rewrites}[1]{\xlongrightarrow{\ #1\ }}
\newcommand{\rwrts}{\rewrites{}}
\newcommand{\redex}[1]{\underline{#1}}

% OCaml rewriting
\newcommand{\oredex}[1]{\underline{#1}}
\newcommand{\orwrts}{\(\rwrts\)}

% better spacing in conditional probabilities, instead of
% \newcommand{\given}{|}           % too little space
%\newcommand{\given}{\mid}     % too much space
\newcommand{\given}{\,|\,}      % just right

% values of random variables
\newcommand{\rvval}[1]{\mathsf{#1}}

% indicator functions
\newcommand{\indic}[1]{\mathbb{I}\left[#1\right]}

% start and end of string symbols
\newcommand{\strbeg}{\mbox{\guillemotleft}}
\newcommand{\strend}{\mbox{\guillemotright}}

% different types of symbols
\newcommand{\sym}[1]{\mbox{#1}}
% nonterminals and terminals
\newcommand{\ntm}[1]{\mbox{\textit{#1}}}
\newcommand{\tm}[1]{\mbox{\textrm{#1}}}
\newcommand{\term}[1]{\ntm{#1}} % deprecated; use \nt
\newcommand{\func}[1]{\operatorname{\mathit{#1}}}

% grammar rules
\newcommand{\goesto}{\rightarrow}

% punctuation in equations
\newcommand{\eqpunc}[1]{{\makebox[0pt][l]{\qquad\rm{#1}}}}

% first mentions and other marginalia
\newcommand{\firstusefont}[1]{\textsc{#1}}
\newcommand{\firstuse}[1]{\firstusemarked{#1}{#1}}
% first use with separate margin and inline (= index) forms
\newcommand{\firstusemarked}[2]{\firstusemarkedindexed{#1}{#2}{#1}}
% first use with separate margin, inline, and index forms
%   for placing the index item at a separate location in the index, use 
%   \firstusemarked{margin}{inline}{index@indexloc}, e.g.,
%   \firstusemarked{$n$-gram}{$n$-gram}{n-gram@$n$-gram}
\newcommand{\firstusemarkedindexed}[3]{\index{#3}\firstusefont{#2}%
  \marginpar[\begin{spacing}{.75}\vspace{-.75ex}\raggedleft\scriptsize\firstusefont{\raggedleft\color{\highlightcolor} #1}\end{spacing}\vspace*{2.75ex}]{\begin{spacing}{.75}\vspace{-.75ex}\raggedright\scriptsize\firstusefont{\raggedright\color{\highlightcolor} #1}\end{spacing}\vspace*{2.75ex}}}


% instructor material
\newif\ifinstructor
\instructortrue
\instructorfalse

\newcommand{\instructor}[1]{\ifinstructor\begin{instructorchunk}#1\end{instructorchunk}\fi}
\newenvironment{instructorchunk}%
{\marginpar[\raggedleft\color{red}\PencilRight]{\raggedright\color{red}\PencilLeft}\color{red}}{}

\newcommand{\inclass}[2]{\ifinstructor\begin{instructorchunk}\begin{description}\item[Question]#1\par\item[Answer]#2\end{description}\end{instructorchunk}\fi}


% displaying shell commands and interactive python

\DefineVerbatimEnvironment%
{commandblock}{Verbatim}
{commandchars=\\\{\}, xleftmargin=2\parindent}

\newcommand{\command}[1]{%
\begin{quote}
\begin{alltt}
#1
\end{alltt}
\end{quote}}
\newcommand{\shellcommand}[1]{\command{\$ #1}}

% advanced topics

\usepackage{manfnt}
%\newcommand{\advanced}[1]{\footnote{\manimpossiblecube: #1}}
%\newcommand{\advanced}[1]{\begin{quote}\color{\lowlightcolor}\manimpossiblecube~\small
%    #1\end{quote}}

%\newenvironment{advanced}{\begin{quote}\color{\lowlightcolor}\manimpossiblecube~\small}{\end{quote}}

\newenvironment{advanced}{\begin{list}{}{%
      \setlength{\leftmargin}{15ex}%
      \setlength{\rightmargin}{0ex}%
    }
  \item \color{\highlightcolor}\manimpossiblecube\hspace{2ex}\small \color{\lowlightcolor} }{\end{list}}

%\renewenvironment{advanced}{\par\begingroup\color{gray}\makebox[0pt][r]{\manimpossiblecube\hspace{5ex}}\small}{\endgroup\par}

% bibliography
\usepackage[colon]{natbib}

% theorems
\newtheorem{theorem}{Theorem}

% problems and exercises
\newcommand{\probmark}{\hfill{$\Box$}}
\newtheorem{probleminternal}{Problem}
\newtheorem{questioninternal}{Question}
\newtheorem{exerciseinternal}[probleminternal]{Exercise}% 
\newtheorem{stageinternal}[probleminternal]{Stage}% 
\newenvironment{exercise}{\begin{exerciseinternal}}{\probmark\end{exerciseinternal}}
\newenvironment{problem}{\begin{probleminternal}}{\probmark\end{probleminternal}}
\newenvironment{stage}{\begin{stageinternal}}{\probmark\end{stageinternal}}
\newenvironment{question}{\begin{questioninternal}}{\probmark\end{questioninternal}}
\newtheorem{definition}{Definition}

\newcommand{\nextproblem}{\vspace{2em}}

\newcommand{\solution}[1]{{\color{\highlightcolor}
   \begin{quote}\textbf{Solution:} #1\end{quote}
   }}
 
% temporary notes
\newcommand{\note}[1]{{\color{red}{[[[#1]]]}}}
%\newcommand{\note}[1]{{\color{red} \textbf{[[[#1]]]}}}
\renewcommand{\note}[1]{}

\sloppy

% line numbering
% Don't need lineno since minted loads it.
%\usepackage[switch*,mathlines]{lineno}
\runningpagewiselinenumbers
\setpagewiselinenumbers
\switchlinenumbers*
\linenumberdisplaymath
\modulolinenumbers[5]
\renewcommand\linenumberfont{\normalfont\tiny\sffamily\color{Tan}}

\usepackage{bussproofs}
\newcommand{\SideCond}[1]{\RightLabel{\ #1}}

\newcommand{\types}{\mathcal{T}}
\newcommand{\consts}[1]{\mathcal{C}_{#1}}
\newcommand{\vars}[1]{\mathcal{X}_{#1}}
\newcommand{\fto}[2]{\langle #1, #2 \rangle}
\newcommand{\domain}[2]{\mathcal{D}_{#1,#2}}
\newcommand{\true}{\mathsf{True}}
\newcommand{\false}{\mathsf{False}}
\newcommand{\model}{\mathcal{M}}
\newcommand{\assign}{\mathcal{G}}
\newcommand{\dennoparams}[1]{\llbracket #1 \rrbracket}
\newcommand{\den}[3]{\dennoparams{#1}_{#2,#3}}
\newcommand{\denmg}[1]{\den{#1}{\model}{\assign}}



\title{CS 51\\ Code Review 1}
\date{\today}
\author{Samuel Green and Gabbi Merz}


\usepackage{fancyvrb,color}

\makeatletter
\def\PY@reset{\let\PY@it=\relax \let\PY@bf=\relax%
    \let\PY@ul=\relax \let\PY@tc=\relax%
    \let\PY@bc=\relax \let\PY@ff=\relax}
\def\PY@tok#1{\csname PY@tok@#1\endcsname}
\def\PY@toks#1+{\ifx\relax#1\empty\else%
    \PY@tok{#1}\expandafter\PY@toks\fi}
\def\PY@do#1{\PY@bc{\PY@tc{\PY@ul{%
    \PY@it{\PY@bf{\PY@ff{#1}}}}}}}
\def\PY#1#2{\PY@reset\PY@toks#1+\relax+\PY@do{#2}}

\expandafter\def\csname PY@tok@ch\endcsname{\let\PY@it=\textit\def\PY@tc##1{\textcolor[rgb]{1.00,0.00,0.00}{##1}}}
\expandafter\def\csname PY@tok@cm\endcsname{\let\PY@it=\textit\def\PY@tc##1{\textcolor[rgb]{1.00,0.00,0.00}{##1}}}
\expandafter\def\csname PY@tok@cs\endcsname{\let\PY@it=\textit\def\PY@tc##1{\textcolor[rgb]{1.00,0.00,0.00}{##1}}}
\expandafter\def\csname PY@tok@cp\endcsname{\let\PY@it=\textit\def\PY@tc##1{\textcolor[rgb]{1.00,0.00,0.00}{##1}}}
\expandafter\def\csname PY@tok@s2\endcsname{\def\PY@tc##1{\textcolor[rgb]{0.00,0.61,0.00}{##1}}}
\expandafter\def\csname PY@tok@s1\endcsname{\def\PY@tc##1{\textcolor[rgb]{0.00,0.61,0.00}{##1}}}
\expandafter\def\csname PY@tok@nc\endcsname{\def\PY@tc##1{\textcolor[rgb]{0.00,0.46,0.46}{##1}}}
\expandafter\def\csname PY@tok@nd\endcsname{\def\PY@tc##1{\textcolor[rgb]{0.80,0.00,0.64}{##1}}}
\expandafter\def\csname PY@tok@si\endcsname{\def\PY@tc##1{\textcolor[rgb]{0.00,0.61,0.00}{##1}}}
\expandafter\def\csname PY@tok@nf\endcsname{\def\PY@tc##1{\textcolor[rgb]{0.76,0.31,0.00}{##1}}}
\expandafter\def\csname PY@tok@sh\endcsname{\def\PY@tc##1{\textcolor[rgb]{0.00,0.61,0.00}{##1}}}
\expandafter\def\csname PY@tok@c1\endcsname{\let\PY@it=\textit\def\PY@tc##1{\textcolor[rgb]{1.00,0.00,0.00}{##1}}}
\expandafter\def\csname PY@tok@kc\endcsname{\def\PY@tc##1{\textcolor[rgb]{0.00,0.00,1.00}{##1}}}
\expandafter\def\csname PY@tok@c\endcsname{\let\PY@it=\textit\def\PY@tc##1{\textcolor[rgb]{1.00,0.00,0.00}{##1}}}
\expandafter\def\csname PY@tok@sx\endcsname{\def\PY@tc##1{\textcolor[rgb]{0.00,0.61,0.00}{##1}}}
\expandafter\def\csname PY@tok@kd\endcsname{\def\PY@tc##1{\textcolor[rgb]{0.00,0.00,1.00}{##1}}}
\expandafter\def\csname PY@tok@ss\endcsname{\def\PY@tc##1{\textcolor[rgb]{0.00,0.61,0.00}{##1}}}
\expandafter\def\csname PY@tok@sr\endcsname{\def\PY@tc##1{\textcolor[rgb]{0.00,0.61,0.00}{##1}}}
\expandafter\def\csname PY@tok@k\endcsname{\def\PY@tc##1{\textcolor[rgb]{0.00,0.00,1.00}{##1}}}
\expandafter\def\csname PY@tok@kn\endcsname{\def\PY@tc##1{\textcolor[rgb]{0.00,0.00,1.00}{##1}}}
\expandafter\def\csname PY@tok@cpf\endcsname{\let\PY@it=\textit\def\PY@tc##1{\textcolor[rgb]{1.00,0.00,0.00}{##1}}}
\expandafter\def\csname PY@tok@kr\endcsname{\def\PY@tc##1{\textcolor[rgb]{0.00,0.00,1.00}{##1}}}
\expandafter\def\csname PY@tok@s\endcsname{\def\PY@tc##1{\textcolor[rgb]{0.00,0.61,0.00}{##1}}}
\expandafter\def\csname PY@tok@kp\endcsname{\def\PY@tc##1{\textcolor[rgb]{0.00,0.00,1.00}{##1}}}
\expandafter\def\csname PY@tok@kt\endcsname{\def\PY@tc##1{\textcolor[rgb]{0.00,0.00,1.00}{##1}}}
\expandafter\def\csname PY@tok@sc\endcsname{\def\PY@tc##1{\textcolor[rgb]{0.00,0.61,0.00}{##1}}}
\expandafter\def\csname PY@tok@sb\endcsname{\def\PY@tc##1{\textcolor[rgb]{0.00,0.61,0.00}{##1}}}
\expandafter\def\csname PY@tok@se\endcsname{\def\PY@tc##1{\textcolor[rgb]{0.00,0.61,0.00}{##1}}}
\expandafter\def\csname PY@tok@sd\endcsname{\def\PY@tc##1{\textcolor[rgb]{0.00,0.61,0.00}{##1}}}

\def\PYZbs{\char`\\}
\def\PYZus{\char`\_}
\def\PYZob{\char`\{}
\def\PYZcb{\char`\}}
\def\PYZca{\char`\^}
\def\PYZam{\char`\&}
\def\PYZlt{\char`\<}
\def\PYZgt{\char`\>}
\def\PYZsh{\char`\#}
\def\PYZpc{\char`\%}
\def\PYZdl{\char`\$}
\def\PYZhy{\char`\-}
\def\PYZsq{\char`\'}
\def\PYZdq{\char`\"}
\def\PYZti{\char`\~}
% for compatibility with earlier versions
\def\PYZat{@}
\def\PYZlb{[}
\def\PYZrb{]}
\makeatother
\begin{document}
\maketitle

\section{Topics}

\begin{itemize}

  \item Record Types and Algebraic Data Types
  \item Currying and Partial Application*
  \item Option Types and Error Handling* 
  \item From last week:
    \begin{itemize}
      \item Higher order functions
      \item Anonymous functions
      \item General syntax: \texttt{match}, \texttt{let}, \texttt{in}
    \end{itemize}
\end{itemize}

\section{Records}

Records allow us to combine related pieces of 
data into one unified structure. Imagine
we wanted to model data for a student. 
A record could be a good choice, because
there are multiple different views of different
types, and they don't have a natural ordering:

\begin{quote}
\begin{Verbatim}[commandchars=\\\{\}]
\PY{o}{\PYZsh{}} \PY{k}{type} \PY{n}{student} \PY{o}{=} \PY{o}{\PYZob{}}
  \PY{n}{name} \PY{o}{:} \PY{k+kt}{string}\PY{o}{;} 
  \PY{n}{email} \PY{o}{:} \PY{k+kt}{string} \PY{o}{;}
  \PY{n}{huid} \PY{o}{:} \PY{k+kt}{int} \PY{o}{;}
  \PY{n}{enrolled} \PY{o}{:} \PY{k+kt}{bool} 
\PY{o}{\PYZcb{}} \PY{o}{;;}
\end{Verbatim}
\end{quote}

Here's an example of a value of this type:

\begin{quote}
\begin{Verbatim}[commandchars=\\\{\}]
\PY{o}{\PYZsh{}} \PY{k}{let} \PY{n}{sam} \PY{o}{=} \PY{o}{\PYZob{}}
  \PY{n}{name} \PY{o}{=} \PY{l+s+s2}{\PYZdq{}}\PY{l+s+s2}{samuel}\PY{l+s+s2}{\PYZdq{}}\PY{o}{;} 
  \PY{n}{email} \PY{o}{=} \PY{l+s+s2}{\PYZdq{}}\PY{l+s+s2}{s@muel.green}\PY{l+s+s2}{\PYZdq{}}\PY{o}{;}
  \PY{n}{huid} \PY{o}{=} \PY{l+m+mi}{51515151}\PY{o}{;} 
  \PY{n}{enrolled} \PY{o}{=} \PY{n+nb+bp}{false}
\PY{o}{\PYZcb{}} \PY{o}{;;} 
          \PY{k}{val} \PY{n}{sam} \PY{o}{:} \PY{n}{student} \PY{o}{=}
  \PY{o}{\PYZob{}}\PY{n}{name} \PY{o}{=} \PY{l+s+s2}{\PYZdq{}}\PY{l+s+s2}{samuel}\PY{l+s+s2}{\PYZdq{}}\PY{o}{;} \PY{n}{email} \PY{o}{=} \PY{l+s+s2}{\PYZdq{}}\PY{l+s+s2}{s@muel.green}\PY{l+s+s2}{\PYZdq{}}\PY{o}{;} \PY{n}{huid} \PY{o}{=} \PY{l+m+mi}{51515151}\PY{o}{;}
   \PY{n}{enrolled} \PY{o}{=} \PY{n+nb+bp}{false}\PY{o}{\PYZcb{}}
\end{Verbatim}
\end{quote}

To pull a field out of a record, we can use
pattern matching, field punning, or dot notation.
There isn't a ``right'' choice for which of
these syntaxes should be preferred -- I'd recommend
relying on the one that makes your code the most
readable. 


\subsection{Pattern Matching}

\begin{Verbatim}[commandchars=\\\{\}]
\PY{k}{let} \PY{n}{get\PYZus{}huid1} \PY{o}{(}\PY{n}{stu} \PY{o}{:} \PY{n}{student}\PY{o}{)} \PY{o}{=} 
  \PY{k}{let} \PY{o}{\PYZob{}} \PY{n}{huid} \PY{o}{=} \PY{n}{h}\PY{o}{;} \PY{o}{\PYZus{}} \PY{o}{\PYZcb{}} \PY{o}{=} \PY{n}{stu} \PY{k}{in} \PY{n}{h} \PY{o}{;;} 

\PY{n}{get\PYZus{}huid1} \PY{n}{sam} \PY{o}{;;} 
\end{Verbatim}

In this case, the identifier \texttt{h} is bound
to the value in the \texttt{huid} field of the record.

\subsection{Field Punning}

With a dryly appropriate name, field punning allows
for direct access of record fields by name in 
pattern matches. For example, this is equivalent
to the function above:

\begin{Verbatim}[commandchars=\\\{\}]
\PY{o}{\PYZsh{}} \PY{k}{let} \PY{n}{get\PYZus{}huid2} \PY{o}{(}\PY{n}{stu} \PY{o}{:} \PY{n}{student}\PY{o}{)} \PY{o}{=} 
  \PY{k}{let} \PY{o}{\PYZob{}} \PY{n}{huid} \PY{o}{;} \PY{o}{\PYZus{}} \PY{o}{\PYZcb{}} \PY{o}{=} \PY{n}{stu} \PY{k}{in} \PY{n}{huid} \PY{o}{;;}
\PY{k}{val} \PY{n}{get\PYZus{}huid2} \PY{o}{:} \PY{n}{student} \PY{o}{\PYZhy{}\PYZgt{}} \PY{k+kt}{int} \PY{o}{=} \PY{o}{\PYZlt{}}\PY{k}{fun}\PY{o}{\PYZgt{}}
\PY{o}{\PYZsh{}} \PY{n}{get\PYZus{}huid2} \PY{n}{sam} \PY{o}{;;}
\PY{o}{\PYZhy{}} \PY{o}{:} \PY{k+kt}{int} \PY{o}{=} \PY{l+m+mi}{51515151}
\end{Verbatim}

As with tuples, we can move this destructuring
into the argument: 

\begin{Verbatim}[commandchars=\\\{\}]
\PY{o}{\PYZsh{}} \PY{k}{let} \PY{n}{get\PYZus{}huid3} \PY{o}{(}\PY{o}{\PYZob{}} \PY{n}{huid}\PY{o}{;} \PY{o}{\PYZus{}} \PY{o}{\PYZcb{}} \PY{o}{:} \PY{n}{student}\PY{o}{)} \PY{o}{=} 
  \PY{n}{huid} \PY{o}{;;}
\PY{k}{val} \PY{n}{get\PYZus{}huid3} \PY{o}{:} \PY{n}{student} \PY{o}{\PYZhy{}\PYZgt{}} \PY{k+kt}{int} \PY{o}{=} \PY{o}{\PYZlt{}}\PY{k}{fun}\PY{o}{\PYZgt{}}
\PY{o}{\PYZsh{}} \PY{n}{get\PYZus{}huid3} \PY{n}{sam} \PY{o}{;;}
\PY{o}{\PYZhy{}} \PY{o}{:} \PY{k+kt}{int} \PY{o}{=} \PY{l+m+mi}{51515151}
\end{Verbatim}

\subsection{Dot Notation}

Finally, there's standard dot notation, 
which is probably most appropriate when accessing
only 1 field of a record. In the case of our
running example: 

\begin{Verbatim}[commandchars=\\\{\}]
\PY{o}{\PYZsh{}} \PY{k}{let} \PY{n}{get\PYZus{}huid4} \PY{o}{(}\PY{n}{s} \PY{o}{:} \PY{n}{student}\PY{o}{)} \PY{o}{=} \PY{n}{s}\PY{o}{.}\PY{n}{huid} \PY{o}{;;}
\PY{k}{val} \PY{n}{get\PYZus{}huid4} \PY{o}{:} \PY{n}{student} \PY{o}{\PYZhy{}\PYZgt{}} \PY{k+kt}{int} \PY{o}{=} \PY{o}{\PYZlt{}}\PY{k}{fun}\PY{o}{\PYZgt{}}
\PY{o}{\PYZsh{}} \PY{n}{get\PYZus{}huid4} \PY{n}{sam} \PY{o}{;;}
\PY{o}{\PYZhy{}} \PY{o}{:} \PY{k+kt}{int} \PY{o}{=} \PY{l+m+mi}{51515151}
\end{Verbatim}

\section{Record Types}

Algebraic data types are great, but what
if we want to model a system or some data
that we need to constraint to a fixed set of 
different values? Imagine, for example,
that we wanted to add a field to the student
record to record the student's \texttt{enrollment type}.

We could start by storing this data
as a string, like this: 

\begin{Verbatim}[commandchars=\\\{\}]
\PY{o}{\PYZsh{}} \PY{k}{type} \PY{n}{student\PYZus{}with\PYZus{}type} \PY{o}{=} \PY{o}{\PYZob{}}
  \PY{n}{email} \PY{o}{:} \PY{k+kt}{string} \PY{o}{;}
  \PY{n}{enrollment\PYZus{}type} \PY{o}{:} \PY{k+kt}{string}
\PY{o}{\PYZcb{}} \PY{o}{;;}
\PY{k}{type} \PY{n}{student\PYZus{}with\PYZus{}type} \PY{o}{=} \PY{o}{\PYZob{}} \PY{n}{email} \PY{o}{:} \PY{k+kt}{string}\PY{o}{;} \PY{n}{enrollment\PYZus{}type} \PY{o}{:} \PY{k+kt}{string}\PY{o}{;} \PY{o}{\PYZcb{}}
\end{Verbatim}

By convention, we could then say that a student
is either ``FAS'', ``DCE'', or ``None''.
We can construct an valid value like this: 

\begin{Verbatim}[commandchars=\\\{\}]
\PY{o}{\PYZsh{}} \PY{k}{let} \PY{n}{new\PYZus{}sam} \PY{o}{=} \PY{o}{\PYZob{}} \PY{n}{email} \PY{o}{=} \PY{l+s+s2}{\PYZdq{}}\PY{l+s+s2}{HUID}\PY{l+s+s2}{\PYZdq{}}\PY{o}{;} 
                \PY{n}{enrollment\PYZus{}type} \PY{o}{=} \PY{l+s+s2}{\PYZdq{}}\PY{l+s+s2}{HEAD TF, NOT STUDENT}\PY{l+s+s2}{\PYZdq{}} \PY{o}{\PYZcb{}}\PY{o}{;;}
\PY{k}{val} \PY{n}{new\PYZus{}sam} \PY{o}{:} \PY{n}{student\PYZus{}with\PYZus{}type} \PY{o}{=}
  \PY{o}{\PYZob{}}\PY{n}{email} \PY{o}{=} \PY{l+s+s2}{\PYZdq{}}\PY{l+s+s2}{HUID}\PY{l+s+s2}{\PYZdq{}}\PY{o}{;} \PY{n}{enrollment\PYZus{}type} \PY{o}{=} \PY{l+s+s2}{\PYZdq{}}\PY{l+s+s2}{HEAD TF, NOT STUDENT}\PY{l+s+s2}{\PYZdq{}}\PY{o}{\PYZcb{}}
\end{Verbatim}

Wait a second! This doesn't meet our convention
for values of the \texttt{student_with_type} type,
but it's still valid, according to \texttt{OCaml}.

\begin{Verbatim}[commandchars=\\\{\}]
\PY{o}{\PYZsh{}} \PY{n}{new\PYZus{}sam} \PY{o}{;;}
\PY{o}{\PYZhy{}} \PY{o}{:} \PY{n}{student\PYZus{}with\PYZus{}type} \PY{o}{=}
\PY{o}{\PYZob{}}\PY{n}{email} \PY{o}{=} \PY{l+s+s2}{\PYZdq{}}\PY{l+s+s2}{HUID}\PY{l+s+s2}{\PYZdq{}}\PY{o}{;} \PY{n}{enrollment\PYZus{}type} \PY{o}{=} \PY{l+s+s2}{\PYZdq{}}\PY{l+s+s2}{HEAD TF, NOT STUDENT}\PY{l+s+s2}{\PYZdq{}}\PY{o}{\PYZcb{}}
\end{Verbatim}

We solve this problem by with Algebraic Data Types (ADTs),
which limit the space of possible values. There
are two types of ADTs: Cartesian products (i.e. tuples)
and sum types (called variants). Here's an example
that uses both: 

\begin{Verbatim}[commandchars=\\\{\}]
\PY{o}{\PYZsh{}} \PY{k}{type} \PY{n}{enrollment} \PY{o}{=} \PY{n+nc}{Faculty} \PY{k}{of} \PY{k+kt}{string} \PY{o}{*} \PY{k+kt}{int} \PY{o}{|} 
                  \PY{n+nc}{FAS} \PY{o}{|} \PY{n+nc}{DCE} \PY{o}{|} \PY{n+nc}{Not} \PY{o}{;;}
\PY{k}{type} \PY{n}{enrollment} \PY{o}{=} \PY{n+nc}{Faculty} \PY{k}{of} \PY{k+kt}{string} \PY{o}{*} \PY{k+kt}{int} \PY{o}{|} \PY{n+nc}{FAS} \PY{o}{|} \PY{n+nc}{DCE} \PY{o}{|} \PY{n+nc}{Not}
\end{Verbatim}

In this case, the \texttt{Faculty} variant is
used to construct named tuples of a string and an int.
The other variant create values that represent those
values distinct, such as: 

\begin{Verbatim}[commandchars=\\\{\}]
\PY{o}{\PYZsh{}} \PY{k}{let} \PY{n}{shieb} \PY{o}{:} \PY{n}{enrollment} \PY{o}{=} \PY{n+nc}{Faculty} \PY{o}{(}\PY{l+s+s2}{\PYZdq{}}\PY{l+s+s2}{Prof. Shieber}\PY{l+s+s2}{\PYZdq{}}\PY{o}{,} \PY{l+m+mi}{1}\PY{o}{)} \PY{o}{;;}
\PY{k}{val} \PY{n}{shieb} \PY{o}{:} \PY{n}{enrollment} \PY{o}{=} \PY{n+nc}{Faculty} \PY{o}{(}\PY{l+s+s2}{\PYZdq{}}\PY{l+s+s2}{Prof. Shieber}\PY{l+s+s2}{\PYZdq{}}\PY{o}{,} \PY{l+m+mi}{1}\PY{o}{)}
\PY{o}{\PYZsh{}} \PY{k}{let} \PY{n}{sam} \PY{o}{:} \PY{n}{enrollment} \PY{o}{=} \PY{n+nc}{FAS} \PY{o}{;;}
\PY{k}{val} \PY{n}{sam} \PY{o}{:} \PY{n}{enrollment} \PY{o}{=} \PY{n+nc}{FAS}
\PY{o}{\PYZsh{}} \PY{k}{let} \PY{n}{reagan} \PY{o}{:} \PY{n}{enrollment} \PY{o}{=} \PY{n+nc}{DCE} \PY{o}{;;} 
\PY{k}{val} \PY{n}{reagan} \PY{o}{:} \PY{n}{enrollment} \PY{o}{=} \PY{n+nc}{DCE}
\PY{o}{\PYZsh{}} \PY{k}{let} \PY{n}{goodell} \PY{o}{:} \PY{n}{enrollment} \PY{o}{=} \PY{n+nc}{Not} \PY{o}{;;}
\PY{k}{val} \PY{n}{goodell} \PY{o}{:} \PY{n}{enrollment} \PY{o}{=} \PY{n+nc}{Not}
\end{Verbatim}

We can embed ADTs in records, name records in ADTs, 
and take advantage of all the other value manipulation
syntax (\texttt{match}, \texttt{let}, type signatures)
that we've see so far. Example:

\begin{Verbatim}[commandchars=\\\{\}]
\PY{o}{\PYZsh{}} \PY{k}{let} \PY{n}{is\PYZus{}shieber} \PY{o}{(}\PY{n}{enr} \PY{o}{:} \PY{n}{enrollment}\PY{o}{)} \PY{o}{:} \PY{k+kt}{bool} \PY{o}{=}
  \PY{k}{match} \PY{n}{enr} \PY{k}{with}
  \PY{o}{|} \PY{n+nc}{Faculty} \PY{o}{(}\PY{n}{name}\PY{o}{,} \PY{o}{\PYZus{}}\PY{o}{)} \PY{o}{\PYZhy{}\PYZgt{}} \PY{n}{name} \PY{o}{=} \PY{l+s+s2}{\PYZdq{}}\PY{l+s+s2}{Prof. Shieber}\PY{l+s+s2}{\PYZdq{}}
  \PY{o}{|} \PY{o}{\PYZus{}} \PY{o}{\PYZhy{}\PYZgt{}} \PY{n+nb+bp}{false} \PY{o}{;;}
\PY{k}{val} \PY{n}{is\PYZus{}shieber} \PY{o}{:} \PY{n}{enrollment} \PY{o}{\PYZhy{}\PYZgt{}} \PY{k+kt}{bool} \PY{o}{=} \PY{o}{\PYZlt{}}\PY{k}{fun}\PY{o}{\PYZgt{}}
\PY{o}{\PYZsh{}} \PY{n}{is\PYZus{}shieber} \PY{n}{shieb} \PY{o}{;;}
\PY{o}{\PYZhy{}} \PY{o}{:} \PY{k+kt}{bool} \PY{o}{=} \PY{n+nb+bp}{true}
\end{Verbatim}

\subsection{Recursive ADTs}

A final note is that ADTs can be recursive.
For example: 
\begin{Verbatim}[commandchars=\\\{\}]
\PY{o}{\PYZsh{}} \PY{k}{type} \PY{n}{partner} \PY{o}{=} \PY{n+nc}{Two} \PY{k}{of} \PY{n}{student} \PY{o}{*} \PY{n}{partner}  \PY{o}{|} \PY{n+nc}{One} \PY{k}{of} \PY{n}{student} \PY{o}{;;}
\PY{k}{type} \PY{n}{partner} \PY{o}{=} \PY{n+nc}{Two} \PY{k}{of} \PY{n}{student} \PY{o}{*} \PY{n}{partner} \PY{o}{|} \PY{n+nc}{One} \PY{k}{of} \PY{n}{student}
\end{Verbatim}
This would allow us to represent students partnerships
and also demonstrates how to use records in ADTs.  

\section{Exercises}

\begin{exercise} Improve the original definition
of \texttt{student_with_type} data type. Why
is the old version bad and the new version better?
\end{exercise}

\ifsoln
\begin{Verbatim}[commandchars=\\\{\}]
\PY{o}{\PYZsh{}} \PY{k}{type} \PY{n}{student} \PY{o}{=} \PY{o}{\PYZob{}}
  \PY{n}{email} \PY{o}{:} \PY{k+kt}{string} \PY{o}{;}
  \PY{n}{enrollment} \PY{o}{:} \PY{n}{enrollment} \PY{o}{;}
\PY{o}{\PYZcb{}} \PY{o}{;;}
\PY{k}{type} \PY{n}{student} \PY{o}{=} \PY{o}{\PYZob{}} \PY{n}{email} \PY{o}{:} \PY{k+kt}{string}\PY{o}{;} \PY{n}{enrollment} \PY{o}{:} \PY{n}{enrollment}\PY{o}{;} \PY{o}{\PYZcb{}}
\PY{o}{\PYZsh{}} \PY{o}{\PYZob{}} \PY{n}{email} \PY{o}{=} \PY{l+s+s2}{\PYZdq{}}\PY{l+s+s2}{s@muel.green}\PY{l+s+s2}{\PYZdq{}}\PY{o}{;} \PY{n}{enrollment} \PY{o}{=} \PY{n+nc}{FAS}\PY{o}{\PYZcb{}} \PY{o}{;;}
\PY{o}{\PYZhy{}} \PY{o}{:} \PY{n}{student} \PY{o}{=} \PY{o}{\PYZob{}}\PY{n}{email} \PY{o}{=} \PY{l+s+s2}{\PYZdq{}}\PY{l+s+s2}{s@muel.green}\PY{l+s+s2}{\PYZdq{}}\PY{o}{;} \PY{n}{enrollment} \PY{o}{=} \PY{n+nc}{FAS}\PY{o}{\PYZcb{}}
\end{Verbatim}
\fi

\begin{exercise} Write a data type \texttt{counts}
to represent counting by one.
Then write a function that
takes a number and returns its \texttt{int} representation.
\end{exercise}

\ifsoln
\begin{Verbatim}[commandchars=\\\{\}]
\PY{o}{\PYZsh{}} \PY{k}{type} \PY{n}{count} \PY{o}{=} \PY{n+nc}{AddOne} \PY{k}{of} \PY{n}{count} \PY{o}{|} \PY{n+nc}{Zero} \PY{o}{;;}
\PY{k}{type} \PY{n}{count} \PY{o}{=} \PY{n+nc}{AddOne} \PY{k}{of} \PY{n}{count} \PY{o}{|} \PY{n+nc}{Zero}
\PY{o}{\PYZsh{}} \PY{k}{let} \PY{k}{rec} \PY{n}{int\PYZus{}of\PYZus{}count} \PY{n}{c} \PY{o}{=} 
  \PY{k}{match} \PY{n}{c} \PY{k}{with}
  \PY{o}{|} \PY{n+nc}{Zero} \PY{o}{\PYZhy{}\PYZgt{}} \PY{l+m+mi}{0}
  \PY{o}{|} \PY{n+nc}{AddOne} \PY{n}{c\PYZsq{}} \PY{o}{\PYZhy{}\PYZgt{}} \PY{l+m+mi}{1} \PY{o}{+} \PY{o}{(}\PY{n}{int\PYZus{}of\PYZus{}count} \PY{n}{c\PYZsq{}}\PY{o}{)} \PY{o}{;;}
\PY{k}{val} \PY{n}{int\PYZus{}of\PYZus{}count} \PY{o}{:} \PY{n}{count} \PY{o}{\PYZhy{}\PYZgt{}} \PY{k+kt}{int} \PY{o}{=} \PY{o}{\PYZlt{}}\PY{k}{fun}\PY{o}{\PYZgt{}}
\PY{o}{\PYZsh{}} \PY{n}{int\PYZus{}of\PYZus{}count} \PY{o}{(}\PY{n+nc}{AddOne}\PY{o}{(}\PY{n+nc}{AddOne}\PY{o}{(}\PY{n+nc}{Zero}\PY{o}{)}\PY{o}{)}\PY{o}{)} \PY{o}{=} \PY{l+m+mi}{2}\PY{o}{;;}
\PY{o}{\PYZhy{}} \PY{o}{:} \PY{k+kt}{bool} \PY{o}{=} \PY{n+nb+bp}{true}
\end{Verbatim}
\fi

\begin{exercise}

Here's the solution to \texttt{curry}. Re-write it 
without using the syntactic sugar of multiple arguments,
and give the complete type of the identifier \texttt{curry}.
\end{exercise}

\ifsoln
\begin{Verbatim}[commandchars=\\\{\}]
\PY{o}{\PYZsh{}} \PY{k}{let} \PY{n}{curry} \PY{n}{f} \PY{n}{x} \PY{n}{y} \PY{o}{=} \PY{n}{f} \PY{o}{(}\PY{n}{x}\PY{o}{,} \PY{n}{y}\PY{o}{)} \PY{o}{;;}
\PY{k}{val} \PY{n}{curry} \PY{o}{:} \PY{o}{(}\PY{k}{\PYZsq{}}\PY{n}{a} \PY{o}{*} \PY{k}{\PYZsq{}}\PY{n}{b} \PY{o}{\PYZhy{}\PYZgt{}} \PY{k}{\PYZsq{}}\PY{n}{c}\PY{o}{)} \PY{o}{\PYZhy{}\PYZgt{}} \PY{k}{\PYZsq{}}\PY{n}{a} \PY{o}{\PYZhy{}\PYZgt{}} \PY{k}{\PYZsq{}}\PY{n}{b} \PY{o}{\PYZhy{}\PYZgt{}} \PY{k}{\PYZsq{}}\PY{n}{c} \PY{o}{=} \PY{o}{\PYZlt{}}\PY{k}{fun}\PY{o}{\PYZgt{}}
\end{Verbatim}
%
\solution{}
If we unwrap the syntactic sugar all the way,
this function can be written as:
%
\begin{Verbatim}[commandchars=\\\{\}]
\PY{o}{\PYZsh{}} \PY{k}{let} \PY{n}{curry} \PY{o}{:} \PY{o}{(}\PY{k}{\PYZsq{}}\PY{n}{a} \PY{o}{*} \PY{k}{\PYZsq{}}\PY{n}{b} \PY{o}{\PYZhy{}\PYZgt{}} \PY{k}{\PYZsq{}}\PY{n}{c}\PY{o}{)} \PY{o}{\PYZhy{}\PYZgt{}} \PY{o}{(}\PY{k}{\PYZsq{}}\PY{n}{a} \PY{o}{\PYZhy{}\PYZgt{}} \PY{k}{\PYZsq{}}\PY{n}{b} \PY{o}{\PYZhy{}\PYZgt{}} \PY{k}{\PYZsq{}}\PY{n}{c}\PY{o}{)} \PY{o}{=} 
  \PY{k}{fun} \PY{n}{f} \PY{o}{\PYZhy{}\PYZgt{}} 
    \PY{k}{fun} \PY{n}{x} \PY{o}{\PYZhy{}\PYZgt{}} 
      \PY{k}{fun} \PY{n}{y} \PY{o}{\PYZhy{}\PYZgt{}} \PY{n}{f} \PY{o}{(}\PY{n}{x}\PY{o}{,} \PY{n}{y}\PY{o}{)} \PY{o}{;;} 
      \PY{k}{val} \PY{n}{curry} \PY{o}{:} \PY{o}{(}\PY{k}{\PYZsq{}}\PY{n}{a} \PY{o}{*} \PY{k}{\PYZsq{}}\PY{n}{b} \PY{o}{\PYZhy{}\PYZgt{}} \PY{k}{\PYZsq{}}\PY{n}{c}\PY{o}{)} \PY{o}{\PYZhy{}\PYZgt{}} \PY{k}{\PYZsq{}}\PY{n}{a} \PY{o}{\PYZhy{}\PYZgt{}} \PY{k}{\PYZsq{}}\PY{n}{b} \PY{o}{\PYZhy{}\PYZgt{}} \PY{k}{\PYZsq{}}\PY{n}{c} \PY{o}{=} \PY{o}{\PYZlt{}}\PY{k}{fun}\PY{o}{\PYZgt{}}
\end{Verbatim}
% 
However, this doesn't seem be the clearest way it can be written. How about expressing it this way: 
\begin{Verbatim}[commandchars=\\\{\}]
\PY{o}{\PYZsh{}} \PY{k}{let} \PY{n}{curry} \PY{o}{:} \PY{o}{(}\PY{k}{\PYZsq{}}\PY{n}{a} \PY{o}{*} \PY{k}{\PYZsq{}}\PY{n}{b} \PY{o}{\PYZhy{}\PYZgt{}} \PY{k}{\PYZsq{}}\PY{n}{c}\PY{o}{)} \PY{o}{\PYZhy{}\PYZgt{}} \PY{o}{(}\PY{k}{\PYZsq{}}\PY{n}{a} \PY{o}{\PYZhy{}\PYZgt{}} \PY{k}{\PYZsq{}}\PY{n}{b} \PY{o}{\PYZhy{}\PYZgt{}} \PY{k}{\PYZsq{}}\PY{n}{c}\PY{o}{)} \PY{o}{=} 
  \PY{k}{fun} \PY{n}{f} \PY{o}{\PYZhy{}\PYZgt{}} 
    \PY{k}{fun} \PY{n}{x} \PY{n}{y} \PY{o}{\PYZhy{}\PYZgt{}} \PY{n}{f} \PY{o}{(}\PY{n}{x}\PY{o}{,} \PY{n}{y}\PY{o}{)} \PY{o}{;;} 
    \PY{k}{val} \PY{n}{curry} \PY{o}{:} \PY{o}{(}\PY{k}{\PYZsq{}}\PY{n}{a} \PY{o}{*} \PY{k}{\PYZsq{}}\PY{n}{b} \PY{o}{\PYZhy{}\PYZgt{}} \PY{k}{\PYZsq{}}\PY{n}{c}\PY{o}{)} \PY{o}{\PYZhy{}\PYZgt{}} \PY{k}{\PYZsq{}}\PY{n}{a} \PY{o}{\PYZhy{}\PYZgt{}} \PY{k}{\PYZsq{}}\PY{n}{b} \PY{o}{\PYZhy{}\PYZgt{}} \PY{k}{\PYZsq{}}\PY{n}{c} \PY{o}{=} \PY{o}{\PYZlt{}}\PY{k}{fun}\PY{o}{\PYZgt{}}
\end{Verbatim}
It's a little clearer here that the function \texttt{curry}
takes a function $f$ of one argument (``uncurried'')
and returns a new function that takes two arguments.
In this case, that function is the anonymous function
\begin{quote}
\begin{Verbatim}[commandchars=\\\{\}]
\PY{k}{fun} \PY{n}{x} \PY{n}{y} \PY{o}{\PYZhy{}\PYZgt{}} \PY{n}{f} \PY{o}{(}\PY{n}{x}\PY{o}{,} \PY{n}{y}\PY{o}{)} \PY{o}{;;} 
\end{Verbatim}
\end{quote}

Going one step farther up makes it into the version
that I find clearest: 

\begin{Verbatim}[commandchars=\\\{\}]
\PY{o}{\PYZsh{}} \PY{k}{let} \PY{n}{curry} \PY{o}{(}\PY{n}{f} \PY{o}{:} \PY{k}{\PYZsq{}}\PY{n}{a} \PY{o}{*} \PY{k}{\PYZsq{}}\PY{n}{b} \PY{o}{\PYZhy{}\PYZgt{}} \PY{k}{\PYZsq{}}\PY{n}{c}\PY{o}{)} \PY{o}{:} \PY{o}{(}\PY{k}{\PYZsq{}}\PY{n}{a} \PY{o}{\PYZhy{}\PYZgt{}} \PY{k}{\PYZsq{}}\PY{n}{b} \PY{o}{\PYZhy{}\PYZgt{}} \PY{k}{\PYZsq{}}\PY{n}{c}\PY{o}{)} \PY{o}{=}
  \PY{k}{fun} \PY{n}{x} \PY{n}{y} \PY{o}{\PYZhy{}\PYZgt{}} \PY{n}{f} \PY{o}{(}\PY{n}{x}\PY{o}{,} \PY{n}{y}\PY{o}{)} \PY{o}{;;}
\PY{k}{val} \PY{n}{curry} \PY{o}{:} \PY{o}{(}\PY{k}{\PYZsq{}}\PY{n}{a} \PY{o}{*} \PY{k}{\PYZsq{}}\PY{n}{b} \PY{o}{\PYZhy{}\PYZgt{}} \PY{k}{\PYZsq{}}\PY{n}{c}\PY{o}{)} \PY{o}{\PYZhy{}\PYZgt{}} \PY{k}{\PYZsq{}}\PY{n}{a} \PY{o}{\PYZhy{}\PYZgt{}} \PY{k}{\PYZsq{}}\PY{n}{b} \PY{o}{\PYZhy{}\PYZgt{}} \PY{k}{\PYZsq{}}\PY{n}{c} \PY{o}{=} \PY{o}{\PYZlt{}}\PY{k}{fun}\PY{o}{\PYZgt{}}
\end{Verbatim}

The important recurring concepts here are:
\begin{enumerate}
  \item There is more than one operationally identical
        way to express the same solution here. Some
        of them are much clearer to read than others!

  \item The type system is our guide throughout. 
\end{enumerate}
\else
\begin{quote}
\begin{Verbatim}[commandchars=\\\{\}]
\PY{k}{let} \PY{n}{curry} \PY{n}{f} \PY{n}{x} \PY{n}{y} \PY{o}{=} \PY{n}{f} \PY{o}{(}\PY{n}{x}\PY{o}{,} \PY{n}{y}\PY{o}{)} \PY{o}{;;}
\end{Verbatim}
\end{quote}
\fi

\begin{exercise} Write a function \texttt{mapfilter}
that takes a list and two functions, $pred$ and $transf$. 
$\texttt{mapfilter}$ should return a list containing
the result of applying \texttt{transf} to every element
for which \texttt{pred} is true, and removing the elements
for which \texttt{pred} is false.

\begin{enumerate}
\item write the type of \texttt{mapfilter}.

\item write \texttt{mapfilter} without using the \texttt{List} module.

\item re-write \texttt{mapfilter} using the \texttt{List}
module. 

\item just for practice, write \texttt{mapfilter} so that
it returns \texttt{None} if the resulting list is empty. Is
this a necessary way to handle errors here? 
\end{enumerate}

\end{exercise}

\ifsoln
\solution{Version 1, without the List module}
\begin{Verbatim}[commandchars=\\\{\}]
\PY{o}{\PYZsh{}} \PY{k}{let} \PY{k}{rec} \PY{n}{mapfilter} \PY{n}{pred} \PY{n}{transf} \PY{n}{lst} \PY{o}{=} 
  \PY{k}{match} \PY{n}{lst} \PY{k}{with}
  \PY{o}{|} \PY{n+nb+bp}{[]} \PY{o}{\PYZhy{}\PYZgt{}} \PY{n+nb+bp}{[]}
  \PY{o}{|} \PY{n}{hd} \PY{o}{::} \PY{n}{tl} \PY{o}{\PYZhy{}\PYZgt{}} \PY{k}{if} \PY{n}{pred} \PY{n}{hd} 
                \PY{k}{then} \PY{o}{(}\PY{n}{transf} \PY{n}{hd}\PY{o}{)} \PY{o}{::} \PY{o}{(}\PY{n}{mapfilter} \PY{n}{pred} \PY{n}{transf} \PY{n}{tl}\PY{o}{)}
                \PY{k}{else} \PY{o}{(}\PY{n}{mapfilter} \PY{n}{pred} \PY{n}{transf} \PY{n}{tl}\PY{o}{)} \PY{o}{;;}
\PY{k}{val} \PY{n}{mapfilter} \PY{o}{:} \PY{o}{(}\PY{k}{\PYZsq{}}\PY{n}{a} \PY{o}{\PYZhy{}\PYZgt{}} \PY{k+kt}{bool}\PY{o}{)} \PY{o}{\PYZhy{}\PYZgt{}} \PY{o}{(}\PY{k}{\PYZsq{}}\PY{n}{a} \PY{o}{\PYZhy{}\PYZgt{}} \PY{k}{\PYZsq{}}\PY{n}{b}\PY{o}{)} \PY{o}{\PYZhy{}\PYZgt{}} \PY{k}{\PYZsq{}}\PY{n}{a} \PY{k+kt}{list} \PY{o}{\PYZhy{}\PYZgt{}} \PY{k}{\PYZsq{}}\PY{n}{b} \PY{k+kt}{list} \PY{o}{=} \PY{o}{\PYZlt{}}\PY{k}{fun}\PY{o}{\PYZgt{}}
\end{Verbatim}

\solution{Version 2, with the List module}
\begin{Verbatim}[commandchars=\\\{\}]
\PY{o}{\PYZsh{}} \PY{k}{let} \PY{n}{mapfilter} \PY{n}{pred} \PY{n}{transf} \PY{n}{lst} \PY{o}{=} 
  \PY{n+nn}{List}\PY{p}{.}\PY{n}{map} \PY{n}{transf} \PY{o}{(}\PY{n+nn}{List}\PY{p}{.}\PY{n}{filter} \PY{n}{pred} \PY{n}{lst}\PY{o}{)} \PY{o}{;;}
\PY{k}{val} \PY{n}{mapfilter} \PY{o}{:} \PY{o}{(}\PY{k}{\PYZsq{}}\PY{n}{a} \PY{o}{\PYZhy{}\PYZgt{}} \PY{k+kt}{bool}\PY{o}{)} \PY{o}{\PYZhy{}\PYZgt{}} \PY{o}{(}\PY{k}{\PYZsq{}}\PY{n}{a} \PY{o}{\PYZhy{}\PYZgt{}} \PY{k}{\PYZsq{}}\PY{n}{b}\PY{o}{)} \PY{o}{\PYZhy{}\PYZgt{}} \PY{k}{\PYZsq{}}\PY{n}{a} \PY{k+kt}{list} \PY{o}{\PYZhy{}\PYZgt{}} \PY{k}{\PYZsq{}}\PY{n}{b} \PY{k+kt}{list} \PY{o}{=} \PY{o}{\PYZlt{}}\PY{k}{fun}\PY{o}{\PYZgt{}}
\end{Verbatim}

\solution{Version 3, just for practice:}
\begin{Verbatim}[commandchars=\\\{\}]
\PY{o}{\PYZsh{}} \PY{k}{let} \PY{n}{mapfilter\PYZus{}option} \PY{n}{pred} \PY{n}{transf} \PY{n}{lst} \PY{o}{=} 
  \PY{k}{let} \PY{n}{res} \PY{o}{=} \PY{n}{mapfilter} \PY{n}{pred} \PY{n}{transf} \PY{n}{lst} \PY{k}{in} 
  \PY{k}{match} \PY{n}{res} \PY{k}{with} 
  \PY{o}{|} \PY{n+nb+bp}{[]} \PY{o}{\PYZhy{}\PYZgt{}} \PY{n+nc}{None}
  \PY{o}{|} \PY{o}{\PYZus{}} \PY{o}{\PYZhy{}\PYZgt{}} \PY{n+nc}{Some} \PY{n}{res} \PY{o}{;;}
\PY{k}{val} \PY{n}{mapfilter\PYZus{}option} \PY{o}{:}
  \PY{o}{(}\PY{k}{\PYZsq{}}\PY{n}{a} \PY{o}{\PYZhy{}\PYZgt{}} \PY{k+kt}{bool}\PY{o}{)} \PY{o}{\PYZhy{}\PYZgt{}} \PY{o}{(}\PY{k}{\PYZsq{}}\PY{n}{a} \PY{o}{\PYZhy{}\PYZgt{}} \PY{k}{\PYZsq{}}\PY{n}{b}\PY{o}{)} \PY{o}{\PYZhy{}\PYZgt{}} \PY{k}{\PYZsq{}}\PY{n}{a} \PY{k+kt}{list} \PY{o}{\PYZhy{}\PYZgt{}} \PY{k}{\PYZsq{}}\PY{n}{b} \PY{k+kt}{list} \PY{n}{option} \PY{o}{=} \PY{o}{\PYZlt{}}\PY{k}{fun}\PY{o}{\PYZgt{}}
\end{Verbatim}
\fi
\end{document}














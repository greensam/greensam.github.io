\usepackage{etex}

% index
%\usepackage{makeidx}   % not needed in AMS styles
\makeindex
\newcommand{\indx}[1]{\index{#1}#1}
\renewcommand{\seename}{see}    % AMS style has it as "see also"

% fonts
\usepackage{palatino}
\usepackage{pxfonts}
\usepackage{pifont}
\usepackage{alltt}
\usepackage{textcomp}
\usepackage{upquote}

% color 
\usepackage[usenames,dvipsnames]{xcolor}

\newcommand{\highlightcolor}{BrickRed}
\newcommand{\lowlightcolor}{Gray}

% listings, links, etc. 
\usepackage[colorlinks=true,
            urlcolor=\highlightcolor,
            linkcolor=\highlightcolor,
            citecolor=\highlightcolor]{hyperref}
\urlstyle{same}  % don't use monospace font for urls 
\usepackage{listings}

% code listings using minted and pygments
\usepackage{minted}
% At some point, need to write a better pygment style as per http://pygments.org/docs/styles/
\usemintedstyle{bw} %trac
\setminted{xleftmargin=2\parindent}
% This environment definition doesn't work, since minted is a verbatim
% environment that needs to see a real \end{minted} not
% a \end{ocamlex}. Too bad.
%\newenvironment{ocamlex}{\begin{minted}{ocaml}}{\end{minted}}


% Configuration for Caml listings using camltex

\newlength{\abovecamlskip}
\setlength{\abovecamlskip}{\bigskipamount}
\newlength{\belowcamlskip}
\setlength{\belowcamlskip}{\bigskipamount}
\newlength{\camlblanklineskip}
\setlength{\camlblanklineskip}{\medskipamount}

% This command allows putting together multiple caml listings with
% controllable space between them.
\newcommand{\nextcaml}{%
\vspace{\camlblanklineskip}%
}

\newcommand{\ocaml}{\texttt{OCaml}}

% tables
\usepackage{tabulary}
\usepackage{booktabs}
\usepackage[tableposition=below]{caption}
\usepackage{graphicx}

% fancy verbatim environment, useful for typesetting program output
\usepackage{fancyvrb}

% linguistic examples
\usepackage{lingmacros}

% interline spacing
\usepackage{setspace}
\onehalfspacing
%\doublespacing

% em-dashes (some inspiration taken from ltugboat.sty)
\newcommand{\dashpunctskip}{\hspace{.25em}\relax}
%\newcommand{\dd}{\unskip{\,--\,}\ignorespaces}
%\newcommand{\dd}{\unskip{---}\ignorespaces}
%\newcommand{\dd}{\unskip{ -- }\ignorespaces}
\newcommand{\dd}{\unskip\nobreak\dashpunctskip--\dashpunctskip\ignorespaces}

% if using xelatex, the following synonym works
%\newcommand{\—}{\dd}

% a literal (like a filename) which should be presented verbatim, but
% allow line breaks but no hyphens inserted
\newcommand{\literal}[1]{\nolinkurl{#1}}
\usepackage{underscore}
\newcommand{\code}[1]{\texttt{#1}}

% tree drawing
%\usepackage{parsetree}
\usepackage{tikz}
\usepackage{tikz-qtree}

% document structure
\newcommand{\newsection}[1]{\clearpage\section{#1}}
\newcommand{\topic}[1]{\subsection*{#1}}
\setcounter{tocdepth}{1}

% symbols
\usepackage{amssymb}
\usepackage{bbding}
\usepackage{mathtools}
\usepackage{textcomp}
\usepackage{extarrows}  % extensible arrows

% random symbols
\renewcommand{\tilde}{\textasciitilde}

% sets and sequences
\newcommand{\seq}[1]{{\langle #1 \rangle}}
\newcommand{\set}[1]{{\{ #1 \}}}
\newcommand{\vect}[1]{\mathbf{#1}} % variables over strings and sequences
\renewcommand{\emptyset}{\varnothing}
\newcommand{\emptystr}{\varepsilon}
\newcommand{\generic}[1]{\mbox{$\langle$\textrm{\textit{#1}}$\rangle$}}

% floor and ceiling
\newcommand{\floor}[1]{\left\lfloor #1 \right\rfloor}
\newcommand{\ceil}[1]{\left\lceil #1 \right\rceil}

% guillemets
\usepackage[T1]{fontenc}

% operators
\newcommand{\argmax}{\operatornamewithlimits{argmax}}
\newcommand{\cnt}[1]{\sharp(#1)}
\newcommand{\definedby}{\triangleq}
\newcommand{\Prob}{{\Pr}}
\newcommand{\Prcap}{\widehat\Prob}
\DeclareMathOperator{\Perplex}{Pp}
\newcommand{\Pathprob}{{\Prob}_{path}} 
\newcommand{\blt}{\mathop{\bullet}}

% semiring operations and values
% circled plus and times for semirings
%\newcommand{\vplusinternal}{\ooalign{\hss$\circ$\hss\cr$+$}}
%\newcommand{\vtimesinternal}{\ooalign{\hss$\circ$\hss\cr$\times$}}
\newcommand{\vplus}{\mathrel{\mathchoice{\vplusinternal}{\vplusinternal}{\scriptsize\vplusinternal}{\tiny\vplusinternal}}}
\newcommand{\vtimes}{\mathrel{\mathchoice{\vtimesinternal}{\vtimesinternal}{\scriptsize\vtimesinternal}{\tiny\vtimesinternal}}}
%\newcommand{\vplus}{\oplus}
%\newcommand{\vtimes}{\otimes}
%\newcommand{\vzero}{{\ooalign{\hss$\circ$\hss\cr$0$}}}
%\newcommand{\vone}{{\ooalign{\hss$\circ$\hss\cr$
%\usepackage{fge}
%\newcommand{\vzero}{\fgestruckzero}
%\newcommand{\vone}{\fgestruckone}
%\newcommand{\vzero}{0}
%\newcommand{\vone}{1}

% numerals
\newcommand{\num}[1]{\underline{\ensuremath{#1}}}
\newcommand{\vzero}{\num{0}}
\newcommand{\vone}{\num{1}}
\newcommand{\vplusinternal}{\underline{+}}
\newcommand{\vtimesinternal}{\underline{\times}}

% logic
\newcommand{\Land}{\wedge}
\newcommand{\Lor}{\vee}
\newcommand{\Lnot}{\neg}
\newcommand{\Lif}{\rightharpoonup}
\newcommand{\Liff}{\rightleftharpoons}
\newcommand{\Ltrue}{T}
\newcommand{\Lfalse}{F}

% types
\newcommand{\cross}{\times}
%\newcommand{\to}{\rightarrow}

% lambda calculus
\newcommand{\subst}[3]{#1[#3 \mapsto #2]}   % subst P for x in Q is \subst{Q}{P}{x}
\newcommand{\rewrites}[1]{\xlongrightarrow{\ #1\ }}
\newcommand{\rwrts}{\rewrites{}}
\newcommand{\redex}[1]{\underline{#1}}

% OCaml rewriting
\newcommand{\oredex}[1]{\underline{#1}}
\newcommand{\orwrts}{\(\rwrts\)}

% better spacing in conditional probabilities, instead of
% \newcommand{\given}{|}           % too little space
%\newcommand{\given}{\mid}     % too much space
\newcommand{\given}{\,|\,}      % just right

% values of random variables
\newcommand{\rvval}[1]{\mathsf{#1}}

% indicator functions
\newcommand{\indic}[1]{\mathbb{I}\left[#1\right]}

% start and end of string symbols
\newcommand{\strbeg}{\mbox{\guillemotleft}}
\newcommand{\strend}{\mbox{\guillemotright}}

% different types of symbols
\newcommand{\sym}[1]{\mbox{#1}}
% nonterminals and terminals
\newcommand{\ntm}[1]{\mbox{\textit{#1}}}
\newcommand{\tm}[1]{\mbox{\textrm{#1}}}
\newcommand{\term}[1]{\ntm{#1}} % deprecated; use \nt
\newcommand{\func}[1]{\operatorname{\mathit{#1}}}

% grammar rules
\newcommand{\goesto}{\rightarrow}

% punctuation in equations
\newcommand{\eqpunc}[1]{{\makebox[0pt][l]{\qquad\rm{#1}}}}

% first mentions and other marginalia
\newcommand{\firstusefont}[1]{\textsc{#1}}
\newcommand{\firstuse}[1]{\firstusemarked{#1}{#1}}
% first use with separate margin and inline (= index) forms
\newcommand{\firstusemarked}[2]{\firstusemarkedindexed{#1}{#2}{#1}}
% first use with separate margin, inline, and index forms
%   for placing the index item at a separate location in the index, use 
%   \firstusemarked{margin}{inline}{index@indexloc}, e.g.,
%   \firstusemarked{$n$-gram}{$n$-gram}{n-gram@$n$-gram}
\newcommand{\firstusemarkedindexed}[3]{\index{#3}\firstusefont{#2}%
  \marginpar[\begin{spacing}{.75}\vspace{-.75ex}\raggedleft\scriptsize\firstusefont{\raggedleft\color{\highlightcolor} #1}\end{spacing}\vspace*{2.75ex}]{\begin{spacing}{.75}\vspace{-.75ex}\raggedright\scriptsize\firstusefont{\raggedright\color{\highlightcolor} #1}\end{spacing}\vspace*{2.75ex}}}


% instructor material
\newif\ifinstructor
\instructortrue
\instructorfalse

\newcommand{\instructor}[1]{\ifinstructor\begin{instructorchunk}#1\end{instructorchunk}\fi}
\newenvironment{instructorchunk}%
{\marginpar[\raggedleft\color{red}\PencilRight]{\raggedright\color{red}\PencilLeft}\color{red}}{}

\newcommand{\inclass}[2]{\ifinstructor\begin{instructorchunk}\begin{description}\item[Question]#1\par\item[Answer]#2\end{description}\end{instructorchunk}\fi}


% displaying shell commands and interactive python

\DefineVerbatimEnvironment%
{commandblock}{Verbatim}
{commandchars=\\\{\}, xleftmargin=2\parindent}

\newcommand{\command}[1]{%
\begin{quote}
\begin{alltt}
#1
\end{alltt}
\end{quote}}
\newcommand{\shellcommand}[1]{\command{\$ #1}}

% advanced topics

\usepackage{manfnt}
%\newcommand{\advanced}[1]{\footnote{\manimpossiblecube: #1}}
%\newcommand{\advanced}[1]{\begin{quote}\color{\lowlightcolor}\manimpossiblecube~\small
%    #1\end{quote}}

%\newenvironment{advanced}{\begin{quote}\color{\lowlightcolor}\manimpossiblecube~\small}{\end{quote}}

\newenvironment{advanced}{\begin{list}{}{%
      \setlength{\leftmargin}{15ex}%
      \setlength{\rightmargin}{0ex}%
    }
  \item \color{\highlightcolor}\manimpossiblecube\hspace{2ex}\small \color{\lowlightcolor} }{\end{list}}

%\renewenvironment{advanced}{\par\begingroup\color{gray}\makebox[0pt][r]{\manimpossiblecube\hspace{5ex}}\small}{\endgroup\par}

% bibliography
\usepackage[colon]{natbib}

% theorems
\newtheorem{theorem}{Theorem}

% problems and exercises
\newcommand{\probmark}{\hfill{$\Box$}}
\newtheorem{probleminternal}{Problem}
\newtheorem{questioninternal}{Question}
\newtheorem{exerciseinternal}[probleminternal]{Exercise}% 
\newtheorem{stageinternal}[probleminternal]{Stage}% 
\newenvironment{exercise}{\begin{exerciseinternal}}{\probmark\end{exerciseinternal}}
\newenvironment{problem}{\begin{probleminternal}}{\probmark\end{probleminternal}}
\newenvironment{stage}{\begin{stageinternal}}{\probmark\end{stageinternal}}
\newenvironment{question}{\begin{questioninternal}}{\probmark\end{questioninternal}}
\newtheorem{definition}{Definition}

\newcommand{\nextproblem}{\vspace{2em}}

\newcommand{\solution}[1]{{\color{\highlightcolor}
   \begin{quote}\textbf{Solution:} #1\end{quote}
   }}
 
% temporary notes
\newcommand{\note}[1]{{\color{red}{[[[#1]]]}}}
%\newcommand{\note}[1]{{\color{red} \textbf{[[[#1]]]}}}
\renewcommand{\note}[1]{}

\sloppy

% line numbering
% Don't need lineno since minted loads it.
%\usepackage[switch*,mathlines]{lineno}
\runningpagewiselinenumbers
\setpagewiselinenumbers
\switchlinenumbers*
\linenumberdisplaymath
\modulolinenumbers[5]
\renewcommand\linenumberfont{\normalfont\tiny\sffamily\color{Tan}}

\usepackage{bussproofs}
\newcommand{\SideCond}[1]{\RightLabel{\ #1}}

\newcommand{\types}{\mathcal{T}}
\newcommand{\consts}[1]{\mathcal{C}_{#1}}
\newcommand{\vars}[1]{\mathcal{X}_{#1}}
\newcommand{\fto}[2]{\langle #1, #2 \rangle}
\newcommand{\domain}[2]{\mathcal{D}_{#1,#2}}
\newcommand{\true}{\mathsf{True}}
\newcommand{\false}{\mathsf{False}}
\newcommand{\model}{\mathcal{M}}
\newcommand{\assign}{\mathcal{G}}
\newcommand{\dennoparams}[1]{\llbracket #1 \rrbracket}
\newcommand{\den}[3]{\dennoparams{#1}_{#2,#3}}
\newcommand{\denmg}[1]{\den{#1}{\model}{\assign}}

